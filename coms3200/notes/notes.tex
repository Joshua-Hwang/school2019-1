\documentclass[a4paper, twoside]{book}

\usepackage{mathtools}
\usepackage{amsmath}
\usepackage{amssymb}
\usepackage{amsfonts}
\usepackage{graphicx}
\usepackage{float}
\usepackage{multirow}
\usepackage{verbatim}

\linespread{1.5}
\setlength{\parindent}{0em}
\setlength{\parskip}{1em}
\setcounter{secnumdepth}{-1}
\setcounter{MaxMatrixCols}{20}
\renewcommand{\arraystretch}{1.5}

\newcommand{\ts}{\textsuperscript}
\newcommand{\diff}{\mathop{}\!\mathrm{d}}
\newcommand{\prob}{\mathbb{P}}
\newcommand{\expect}{\mathbb{E}}

\DeclarePairedDelimiter{\abs}{\lvert}{\rvert}
\DeclarePairedDelimiter\norm{\lVert}{\rVert}
\DeclarePairedDelimiter\p{\lparan}{\rparan}

\title{Lecture notes}
\author{Joshua Hwang (44302650)}

\begin{document}
\maketitle

This document is designed to be an top down view of the network stack.
Additional information must be acquired through alternative means.

\tableofcontents

\chapter{Overview and definitions}
\begin{description}
    \item [hosts] are end systems that are running network apps. Things like
        PCs, servers, wireless laptops and smartphones just to name a few.
    \item [communication links] are thiings like fiber, copper, radio,
        satellite. Different mediums used to engage with the network.
    \item [transmission rate] see bandwith
    \item [packet switches] forwards packets. Some examples include routers
        and link layer switches.
    \item [protocols] define format, order of messages sent and received
        among network entities, and actions taken on message transmission,
        receipt.
    \item [bandwidth] usually in bits per second
\end{description}

\paragraph{Nuts and bolts view}
From a nuts and bolts view,
the internet is a ``network of networks'', a load of interconnected ISPs.
All this communication is handled by protocols. These protocols along with
other internet standards are constructed by the Internet Engineering Task
Force (\textbf{IETF}) which create Requests For Comments (\textbf{RFC}) which
provide technical details of protocols.

\paragraph{Service view}
From a service point of view, the internet provides infrastructure for
numerous services that applications can use. Programmers and app creators can
use these services through hooks (?) that allow apps to send and receive
and ``connect'' to the internet. The internet also provides service options
analogous to postal service options.

You can connect an end system to the internet through several means. These
include residential access nets (home networks), institutional access networks
(school or company) and mobile access networks.

Digital Subscriber Line (DSL) uses existing telephone line to a central
off

\section{Checklist}
Explain the core concepts of the terms used in computer networking.

\chapter{Application Layer}
In this section we covered,
\begin{itemize}
    \item HTTP
    \item SMTP
    \item DNS
    \item P2P
    \item CDN
    \item Sockets
\end{itemize}

\section{Checklist}
Find and explain the detailed information given in a HTTP request/response
and DNS request/response or any of the other mentioned application layer
protocols.

\chapter{Transport Layer}
In this section we covered,
\begin{itemize}
    \item UDP
    \item TCP
\end{itemize}

\section{Checklist}
Explain difference between UDP and TCP/@.

Explain how to guarantee reliable delivery of application messages despite
an underlying, unreliable channel.

Find information in a TCP or UDP dump.

\chapter{Network Layer}
In this section we covered,
\begin{itemize}
    \item Data Plane
    \item Control Plane
\end{itemize}

\section{Checklist}
Explain difference between distance vector and link state routing algorithms.

Run Dijkstra's algorithm and Distance Vector algorithm (Bellman-Form
algorithm).

\chapter{Link Layer}

\section{Checklist}
Explain how parity bits work.

Construct parity bit and also detect errors with parity bit.

Explain 2D parity and its limitations.

Calculate and show how CRC is used to detect errors.

Explain why forward error correction (FEC) is used.

Explain MAC addresses. Find relevant information in a MAC address.

\chapter{Security}

\section{Checklist}
Explain security goals.

Encrypt messages using classical ciphers.

Briefly explain security solutions.

\end{document}
