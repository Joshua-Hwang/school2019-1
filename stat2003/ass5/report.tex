\documentclass{article}
\usepackage{amsmath}
\usepackage{amssymb}
\usepackage{graphicx}
\usepackage{float}
\usepackage{multirow}
\usepackage{verbatim}

\setlength{\parindent}{0em}
\setlength{\parskip}{1em}
\renewcommand{\arraystretch}{1.5}

\newcommand{\diff}{\mathop{}\!\mathrm{d}}
\newcommand{\prob}{\mathbb{P}}
\newcommand{\expect}{\mathbb{E}}

\title{Assignment 5}
\author{Joshua Hwang (44302650)}
\date{13 May}

\begin{document}
\maketitle

\section{Approximate $S_{100}$}
Using the Central Limit Theorem we can approximate $\prob(S_{100} \leq 562)$.
\begin{align*}
    \prob \left( \frac{S_{100} - 100\mu}{\sigma \sqrt{100}} \leq x \right) \\
    &= \prob \left( S_{100} \leq 10 \sigma x + 100\mu \right) &\approx \Phi(x) \\
\end{align*}

We know $\mu = \frac{363}{65} \approx 5.5846$ but we need to find $\sigma$.
We find this by computing $\expect (X - \expect X)^2$.
\begin{align*}
    \text{Var} X &= \int_x (x - \mu)^2 f(x) \diff x \\
    &= \frac{3}{52} \int_1^9 (x - \frac{363}{65})^2 \sqrt{x} \diff x \\
    &= \frac{3}{52} \int_1^9
    \left( x^2 - 2x\frac{363}{65} + \frac{131769}{4225} \right)
    \sqrt{x} \diff x \\
    &= \frac{3}{52} \int_1^9
    x^{2.5} - \frac{726}{65}x^{1.5} + x^{0.5}\frac{131769}{4225} \diff x \\
    &= \frac{3}{52} \left[
    \frac{2}{7}x^{3.5} - \frac{1452}{325}x^{2.5} + x^{1.5}\frac{87846}{4225}
    \right]_1^9 \\
    &= \frac{3}{52} 83.9807 && \text{To four decimal places} \\
    &= 4.8450 \\
    \sigma &= 2.2011
    && \text{Standard deviation is the square root of Variance}
\end{align*}

Now,
\begin{align*}
    \prob \left( \frac{S_{100} - 100\mu}{\sigma \sqrt{100}} \leq x \right) \\
    &= \prob \left( S_{100} \leq 10 \sigma x + 100\mu \right) &\approx \Phi(x) \\
    &= \prob \left( S_{100} \leq 10 \times 2.2011 x + 100 \times 5.5846 \right) \\
    &= \prob \left( S_{100} \leq 22.011 x + 558.46 \right)
\end{align*}

We ensure the right hand side of our probability is 562.
\begin{align*}
    562 &= 22.011 x + 558.46 \\
    x &= 0.1608 \\
\end{align*}

Now we need to find,
\begin{align*}
    \Phi(0.1608) &\approx 0.5636
    &= 5.6 \times 10^{-1}
\end{align*}

We create a Python program to simulate 10000 $S_{100}$ as well.
\verbatiminput{q1.py}

This gave us
\begin{verbatim}
Approx answer 0.5588
\end{verbatim}

Which fits closely to our own result of 0.5636.

\section{Column vector}
\subsection{Write out using vectors}
$Z$ is a $3\times1$ column vector of standard normals.
\begin{align*}
    X &= \mu + BZ \\
    X
    &=
    \begin{bmatrix}
        0 \\ 1 \\ -1
    \end{bmatrix}
    +
    \begin{bmatrix}
        1 & 0 & 0 \\
        0 & 2 & 0 \\
        0 & 0 & \sqrt{2} \\
    \end{bmatrix}
    Z \\
\end{align*}

\subsection{What is the covariance matrix of $X$}
ZZZ - Use the $A (Cov) A^T$ but (Cov) is identity because standard normals
The covariance matrix defined for the $(i,j)$ element is,
\begin{align*}
    \text{Cov}(X_i,X_j) & = \expect (X_i - \mu_i)(X_j - \mu_j) \\
\end{align*}

It is obvious that if $i = j$ we just have variance and the elements
$(i, j) = (j ,i)$. Since all $X_i$ are independent we know
$\expect X_i X_j = \expect X_i \expect X_j$ when $i \neq j$.

Thus for the elements in our covariance matrix we have,
\begin{align*}
    \text{Cov}(X_i,X_j) & = \expect X_i X_j - \expect X_i \expect X_j \\
    & = \expect X_i \expect X_j - \expect X_i \expect X_j \\
    & = 0 \\
\end{align*}

Thus our matrix is now,
\begin{align*}
    \begin{bmatrix}
        1 & 0 & 0 \\
        0 & 4 & 0 \\
        0 & 0 & 2 \\
    \end{bmatrix}
\end{align*}

\subsection{Expectation of $Y_1$}
Find the expectation of $Y_1 = X_1 + X_3$. We will make use of helpful
properties of expectation.
\begin{align*}
    \expect Y_1 &= \expect X_1 + \expect X_3 \\
    &= 0 + -1 \\
    \expect Y_1 &= -1 \\
\end{align*}

\subsection{Distribution of $Y_2$}
Find the distribution of $Y_2 = X_1 + X_2 - 2X_3$. We note that normal
distributions are all affine transformations of the standard normal. Thus,
\begin{align*}
    Y_2 &= X_1 + X_2 - 2X_3 \\
    &= N(0,1) + (1 + 2 N(0,1)) - 2(-1 + \sqrt{2} N(0,1)) \\
    &= 3 + (3 - 2\sqrt{2}) N(0,1) \\
    &= N(3, (3-2\sqrt{2})^2) \\
\end{align*}

Thus $Y_2$ has a normal distribution with $\mu = 3$
and $\sigma = 3 - 2\sqrt{2}$.

\subsection{Give joint distribution of $Y_1$ and $Y_2$}
We consider a column vector $Y = \begin{bmatrix} Y_1 \\ Y_2 \end{bmatrix}$.
From here we attempt to put $Y$ in context of standard normals.
\begin{align*}
    Y &= \begin{bmatrix} Y_1 \\ Y_2 \end{bmatrix} \\
    &= \begin{bmatrix} X_1 & +0 & +X_3 \\ X_1 & +X_2 & -2X_3 \end{bmatrix} \\
    &= \begin{bmatrix} 1 & 0 & 1 \\ 1 & 1 & -2 \end{bmatrix}
    \begin{bmatrix} X_1 \\ X_2 \\ X_3 \end{bmatrix} \\
    &= \begin{bmatrix} 1 & 0 & 1 \\ 1 & 1 & -2 \end{bmatrix}
    \left(
        \begin{bmatrix}
            0 \\ 1 \\ -1
        \end{bmatrix}
        +
        \begin{bmatrix}
            1 & 0 & 0 \\
            0 & 2 & 0 \\
            0 & 0 & \sqrt{2} \\
        \end{bmatrix}
        Z
    \right) \\
    &= \begin{bmatrix} 1 & 0 & 1 \\ 1 & 1 & -2 \end{bmatrix}
    \begin{bmatrix}
        0 \\ 1 \\ -1
    \end{bmatrix}
    +
    \begin{bmatrix} 1 & 0 & 1 \\ 1 & 1 & -2 \end{bmatrix}
    \begin{bmatrix}
        1 & 0 & 0 \\
        0 & 2 & 0 \\
        0 & 0 & \sqrt{2} \\
    \end{bmatrix}
    Z  \\
    &= \begin{bmatrix} -1 \\ 3 \end{bmatrix}
    +
    \begin{bmatrix}
        1 & 0 & \sqrt{2} \\
        1 & 2 & -2\sqrt{2} \\
    \end{bmatrix}
    Z
\end{align*}

From here we can calculate the covariance matrix $\Sigma = BB^T$ and obtain
the pdf for the joint distribution.
\begin{align*}
    \Sigma &= BB^T \\
    &=
    \begin{bmatrix}
        1 & 0 & \sqrt{2} \\
        1 & 2 & -2\sqrt{2} \\
    \end{bmatrix}
    \begin{bmatrix}
        1 & 1 \\
        0 & 2 \\
        \sqrt{2} & -2\sqrt{2}
    \end{bmatrix} \\
    &=
    \begin{bmatrix}
        3 & -3 \\
        -3 & 13
    \end{bmatrix}
\end{align*}

Note the determinant for $\Sigma \neq 0$ thus our matrix is invertible.
\begin{align*}
    \begin{bmatrix}
        3 & -3 \\
        -3 & 13
    \end{bmatrix}^{-1}
    &= \frac{1}{13 \times 3 - 3 \times 3}
    \begin{bmatrix}
        13 & 3 \\
        3 & 3
    \end{bmatrix} \\
    &= \frac{1}{30}
    \begin{bmatrix}
        13 & 3 \\
        3 & 3
    \end{bmatrix} \\
\end{align*}

We know substitute our values into the jointly normal pdf.
($n=2$ is the dimension of $Y$)
\begin{align*}
    f_Y(y) &= \frac{1}{\sqrt{(2\pi)^n det(\Sigma)}}
    e^{-\frac{1}{2} (y-\mu)^T \Sigma^{-1} (y-\mu)} \\
    &= \frac{1}{2\pi\sqrt{30}}
    e^{-\frac{1}{2} (y-\begin{bmatrix} -1 \\ 3 \end{bmatrix})^T
    \frac{1}{30}
    \begin{bmatrix}
        13 & 3 \\
        3 & 3
    \end{bmatrix} (y-\begin{bmatrix} -1 \\ 3 \end{bmatrix})} \\
\end{align*}

\end{document}
