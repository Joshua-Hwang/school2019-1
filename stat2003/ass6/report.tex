\documentclass{article}
\usepackage{amsmath}
\usepackage{amssymb}
\usepackage{graphicx}
\usepackage{float}
\usepackage{multirow}
\usepackage{verbatim}

\setlength{\parindent}{0em}
\setlength{\parskip}{1em}
\renewcommand{\arraystretch}{1.5}

\newcommand{\diff}{\mathop{}\!\mathrm{d}}
\newcommand{\prob}{\mathbb{P}}
\newcommand{\expect}{\mathbb{E}}

\title{Assignment 5}
\author{Joshua Hwang (44302650)}
\date{13 May}

\begin{document}
\maketitle

\section{One-step transition matrix}
Consider the one-step transition matrix,
\[
    P =
    \begin{bmatrix}
        0.1 & 0.1 & 0.8 \\
          0 & 0.6 & 0.4 \\
        0.3 & 0.7 &   0
    \end{bmatrix}
\]

\subsection{$\prob(X_1 = 3 | X_0 = 1)$}
We have the initial distribution as
\[
    \pi_0 =
    \begin{bmatrix}
        1 & 0 & 0
    \end{bmatrix}
\]

Thus we find the distribution at $X_1$,
\begin{align*}
    \pi_1
    &=
    \begin{bmatrix}
        1 & 0 & 0
    \end{bmatrix}
    \begin{bmatrix}
        0.1 & 0.1 & 0.8 \\
          0 & 0.6 & 0.4 \\
        0.3 & 0.7 &   0
    \end{bmatrix} \\
    &=
    \begin{bmatrix}
        0.1 & 0.1 & 0.8
    \end{bmatrix}
\end{align*}

Thus, $\prob(X_1 = 3 | X_0 = 1) = 0.1$ from the first column (which is $X_1$).

\subsection{$\expect[X_1 | X_0 = 1]$}
From the previous question we already have
\[
    \begin{bmatrix}
        0.1 & 0.1 & 0.8
    \end{bmatrix}
\]

Thus we compute, $0.1 \times 1 + 0.1 \times 2 + 0.8 \times 3 = 2.7$.

\subsection{$\prob(X_2 = 1 | X_0 = 2)$}
We apply the one-step transition matrix twice.
\begin{align*}
    \pi_2
    &=
    \begin{bmatrix}
        0 & 1 & 0
    \end{bmatrix}
    \begin{bmatrix}
        0.1 & 0.1 & 0.8 \\
          0 & 0.6 & 0.4 \\
        0.3 & 0.7 &   0
    \end{bmatrix}^2 \\
    &=
    \begin{bmatrix}
        0 & 1 & 0
    \end{bmatrix}
    \begin{bmatrix}
        0.25 & 0.63 & 0.12 \\
        0.12 & 0.64 & 0.24 \\
        0.03 & 0.45 & 0.52
    \end{bmatrix}^2 \\
    &=
    \begin{bmatrix}
        0.12 & 0.64 & 0.24
    \end{bmatrix}
\end{align*}

Thus, $\prob(X_2 = 1 | X_0 = 2) = 0.64$

\subsection{$\prob(X_1 = 1, X_2 = 2, X_3 = 3 | X_0 = 1)$}
Referring to the lecture notes we have the following formula derived from the
product rule and the Markov rule and expansion of conditional probablity.
\begin{align*}
    \prob(X_1 = 1, X_2 = 2, X_3 = 3 | X_0 = 1)
    &= \frac{\prob(X_0 = 1, X_1 = 1, X_2 = 2, X_3 = 3)}{\prob {X_0 = 1}} \\
    &= \frac{\prob(X_0 = 1) \prob(X_1 = 1 | X_0 = 1) \prob(X_2 = 2 | X_1 = 1) \prob(X_3 = 3 | X_2 = 2)}
        {\prob {X_0 = 1}} \\
    &= \prob(X_1 = 1 | X_0 = 1) \prob(X_2 = 2 | X_1 = 1) \prob(X_3 = 3 | X_2 = 2)
\end{align*}

Performing a single multiplication of our one-step transition matrix will move
us from time $i$ to time $i+1$. Thus, e now calculate each probability
in the same way as the first question.
\begin{align*}
    \pi_1
    &=
    \begin{bmatrix}
        1 & 0 & 0
    \end{bmatrix}
    \begin{bmatrix}
        0.1 & 0.1 & 0.8 \\
          0 & 0.6 & 0.4 \\
        0.3 & 0.7 &   0
    \end{bmatrix} \\
    &=
    \begin{bmatrix}
        0.1 & 0.1 & 0.8
    \end{bmatrix}
\end{align*}
Thus, $\prob(X_1 = 1 | X_0 = 1) = 0.1$

\begin{align*}
    \pi_2
    &=
    \begin{bmatrix}
        1 & 0 & 0
    \end{bmatrix}
    \begin{bmatrix}
        0.1 & 0.1 & 0.8 \\
          0 & 0.6 & 0.4 \\
        0.3 & 0.7 &   0
    \end{bmatrix} \\
    &=
    \begin{bmatrix}
        0.1 & 0.1 & 0.8
    \end{bmatrix}
\end{align*}
Thus, $\prob(X_2 = 2 | X_1 = 1) = 0.1$

\begin{align*}
    \pi_3
    &=
    \begin{bmatrix}
        0 & 1 & 0
    \end{bmatrix}
    \begin{bmatrix}
        0.1 & 0.1 & 0.8 \\
          0 & 0.6 & 0.4 \\
        0.3 & 0.7 &   0
    \end{bmatrix} \\
    &=
    \begin{bmatrix}
        0 & 0.6 & 0.4
    \end{bmatrix}
\end{align*}
Thus, $\prob(X_3 = 3 | X_2 = 2) = 0.4$

Note we did not have to perform these calculations since each row vector
was extracting its respective row from the matrix from which we could have
further extracted the relevant cell.

Now we may calculate our final result,
\begin{align*}
    \prob(X_1 = 1 | X_0 = 1) \prob(X_2 = 2 | X_1 = 1) \prob(X_3 = 3 | X_2 = 2)
    &= 0.1 \times 0.1 \times 0.4 \\
    &= 0.004
\end{align*}

Thus, $\prob(X_1 = 1, X_2 = 2, X_3 = 3 | X_0 = 1) = 0.004$

\subsection{The stationary distribution}
The stationary distribution satisfies the following $\pi = \pi P$.
We solve via,
\begin{align*}
    \pi &= \pi P \\
    \pi I &= \pi P \\
    \pi (P - I) &= 0 \\
    (P - I)^T \pi^T &= 0^T \\
\end{align*}

We could either calculate the appropriate vector through Gaussian elimination
or, in this case, use a program to calculate the nullspace (kernel) of our
matrix.
\begin{verbatim}
import numpy as np
from numpy.linalg import svd

def nullspace(A, atol=1e-13, rtol=0):
    A = np.atleast_2d(A)
    u, s, vh = svd(A)
    tol = max(atol, rtol * s[0])
    nnz = (s >= tol).sum()
    ns = vh[nns:].conj().T
    return ns

A = np.matrix('0.1 0.1 0.8; 0 0.6 0.4; 0.3 0.7 0')
p = nullspace(np.transpose(A - np.identity(3)))
p = p/sum(p)
print(p.T)
\end{verbatim}

From this we find the vector (to 4 decimal places)
\[
    \begin{bmatrix}
        0.1053 & 0.5789 & 0.3158
    \end{bmatrix}
\]

\subsection{$\expect[X_2 | X_0 = 2]$}
Recall the section where we calculated $\prob(X_2 = 1 | X_0 = 2)$.
We got the following final vector
\[
    \begin{bmatrix}
        0.12 & 0.64 & 0.24
    \end{bmatrix}
\]

Thus the expectation from this vector is,
$0.12 \times 1 + 0.64 \times 2 + 0.24 \times 3 = 2.12$.

\subsection{$\lim_{n \to \infty} P^n$}
All rows of $P^\infty$ are equal to the limiting distribution $\pi$.
Since we've calculated the limiting distribution already
\[
    \begin{bmatrix}
        0.1053 & 0.5789 & 0.3158 \\
        0.1053 & 0.5789 & 0.3158 \\
        0.1053 & 0.5789 & 0.3158
    \end{bmatrix}
\]

\section{Mouse maze}
\subsection{Determine the one-step transition matrix}
From the assumptions stated we may create a transition matrix.
A one-step transition matrix is constructed from transitions
from a state (columns) to a state (rows). From these we get the following,
\[
    \begin{bmatrix}
        0 & 0.5 & 0.5 & 0 & 0 & 0 \\
        1 & 0 & 0 & 0 & 0 & 0 \\
        0.5 & 0 & 0 & 0.5 & 0 & 0 \\
        0 & 0 & 0.33 & 0 & 0.33 & 0.33 \\
        0 & 0 & 0 & 0.5 & 0 & 0.5 \\
        0 & 0 & 0 & 0.5 & 0.5 & 0 \\
    \end{bmatrix}
\]

Note the $0.33$ is actually $\frac{1}{3}$ I'm just lazy.

\subsection{Calculate a probability}
\begin{verbatim}
ZZZ - please fill in
A = np.matrix('0 0.5 0.5 0 0 0; 1 0 0 0 0 0; 0.5 0 0 0.5 0 0; 0 0 0.333333333333 0 0.3333333333333333333 0.333333333333333333; 0 0 0 0.5 0 0.5; 0 0 0 0.5 0.5 0')
>>> A
matrix([[0.        , 0.5       , 0.5       , 0.        , 0.        ,
         0.        ],
        [1.        , 0.        , 0.        , 0.        , 0.        ,
         0.        ],
        [0.5       , 0.        , 0.        , 0.5       , 0.        ,
         0.        ],
        [0.        , 0.        , 0.33333333, 0.        , 0.33333333,
         0.33333333],
        [0.        , 0.        , 0.        , 0.5       , 0.        ,
         0.5       ],
        [0.        , 0.        , 0.        , 0.5       , 0.5       ,
         0.        ]])
>>> np.matrix('0 0 0 1 0 0')*A**7
matrix([[0.05555556, 0.109375  , 0.2193287 , 0.21875   , 0.19849537,
         0.19849537]])
\end{verbatim}

The answer is 0.19849537

The second answer is 0.1666666666666 which is 1/6
\end{document}
