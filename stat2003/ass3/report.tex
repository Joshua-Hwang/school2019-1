\documentclass{article}
\usepackage{amsmath}
\usepackage{amssymb}
\usepackage{graphicx}
\usepackage{float}

\usepackage{verbatim}

\setlength{\parindent}{0em}
\setlength{\parskip}{1em}

\title{Assignment 3}
\author{Joshua Hwang (44302650)}
\date{30 March}

\begin{document}
\maketitle

\section{White Walkers}
A Poisson distribution is given by,
\begin{align*}
    P(X=k) &= \frac{\lambda^k e^{-\lambda}}{k!}
\end{align*}

A neat property of the Poisson distribution is the mean and variance are
equal. In this particular case, the second moment is $E(X^2)=0.002a$, where
$a$ is the area in square kilometers.

We use the equation for variance to determine our $\lambda$,
\begin{align*}
    Var(X) &= E(X^2) - E(X)^2 \\
    \lambda &= E(X^2) - \lambda^2
    && \text{Both the variance and mean are $\lambda$} \\
    E(X^2) &= \lambda + \lambda^2 && \text{The second moment is $0.002a$} \\
    0.002a &= \lambda + \lambda^2 \\
    0 &= \lambda^2 + \lambda - 0.002a && \text{Solve the quadratic equation} \\
    \lambda &= \frac{-1 \pm \sqrt{1^2 - 4(1)(-0.002a)}}{2(1)} \\
    &= \frac{-1 \pm \sqrt{1 + 0.008a}}{2}
\end{align*}

Since variance can never be negative (since it's the squared deviation from the
mean and the standard deviation would not make sense), we produce the following
probability function with regards to area $a$.
\begin{align*}
    \lambda &= \frac{-1 + \sqrt{1 + 0.008a}}{2}
\end{align*}

\subsection{Expected number of White Walkers}
We're asked for expected number of White Walkers given $a=20\times30=600$.
Since we're using a Poisson distribution our lambda is our expected value.
\begin{align*}
    \lambda &= \frac{-1 + \sqrt{1 + 0.008a}}{2} \\
    &= \frac{-1 + \sqrt{1 + 0.008(600)}}{2} \\
    &= \frac{-1 + \sqrt{1 + 4.8}}{2} \\
    &\approx 0.7042 && \text{To 4 decimal places}
\end{align*}

There is expected to be 0.7042 White Walkers in a 20 by 30 km region.

\subsection{10 White Walkers}
Our new $a$ will be 1 million hectares which is 10000 square kilometres.
\begin{align*}
    \lambda &= \frac{-1 + \sqrt{1 + 0.008a}}{2} \\
    &= \frac{-1 + \sqrt{1 + 0.008(10000)}}{2} \\
    &= \frac{-1 + \sqrt{1 + 80}}{2} \\
    &= 4 
\end{align*}

Thus our PMF is now,
\begin{align*}
    P(X=k) &= \frac{\lambda^k e^{-\lambda}}{k!} \\
    &= \frac{4^k e^{-4}}{k!}
\end{align*}

Now we find the probability of more than 10.
\begin{align*}
    P(X>10) &= 1 - P(X\leq10) \\
    &= 1 - (P(X=0) + P(X=1) + P(X=2) + ... + P(X=9) + P(X=10)) \\
    &= 1 - (0.9972) && \text{To 4 decimal places} \\
    &= 0.0028 \\
\end{align*}

Thus, the probability of having more than 10 White Walkers in
a million hectares is 0.0028.

\section{Fair dice}
This situation can be modelled as a geometric distribution.
\begin{align*}
    P(X=k) &= (1-p)^{k-1}p
\end{align*}

Also note,
\begin{align*}
    P(X \geq k) &= \sum_{i=k}^\infty (1-p)^{i-1}p \\
    &= p \sum_{i=k}^\infty (1-p)^{i-1} && \text{Infinite geometric series} \\
    &= p \frac{(1-p)^{k-1}}{1-(1-p)}
    && \text{Probabilities are less than 1 as long as $p$ is not impossible} \\
    &= p \frac{(1-p)^{k-1}}{p} \\
    &= (1-p)^{k-1} \\
\end{align*}

In our case the probability that any of the dice give a result of 5 or higher
is,
\begin{align*}
    P(\text{any greater or equal to 5}) &= 1 - P(\text{all less than 5}) \\
    &= 1 - \left(\frac{4}{6}\times\frac{4}{6}\times\frac{4}{6}\right) \\
    &= \frac{19}{27} \\
\end{align*}

Thus our PMF is,
\begin{align*}
    P(X=k) &= \left(1-\frac{19}{27}\right)^{k-1}\frac{19}{27} \\
    &= \left(\frac{8}{27}\right)^{k-1}\frac{19}{27}
\end{align*}

\subsection{Third toss}
The probability we stop at or before 3 rolls is determined by,
\begin{align*}
    P(X \leq 3) &= 1 - P(X > 3) \\
    &= 1 - P(X \geq 4) \\
    &= 1 - \left(\left(\frac{8}{27}\right)^{4-1}\right) \\
    &= 0.9740 && \text{To 4 decimal places} \\
    &= P(X=1) + P(X=2) + P(X=3) && \text{An alternative approach...} \\
    &= 0.9740 && \text{To 4 decimal places} \\
\end{align*}

\subsection{More than 10}
We're looking for,
\begin{align*}
    P(X>10|\text{No 6 in 6 throws})
    &= \frac{P(X>10 \cap \text{No 6 in 6 throws})}
    {P(\text{No 6 in 6 throws})} \\
    \intertext{Note: achieving no 6s in 6 throws is guaranteed to occur
    if more than 10 throws occur. Thus 6 throws is a subset of $X>10$.} \\
    &= \frac{P(X>10)}{P(\text{No 6 in 6 throws})} \\
\end{align*}

Now we need to find the probability of not rolling a 6 in 6 rolls. We use the
same techniques as before. The chance that three dice fail to roll a 6 is,
\begin{align*}
    P(\text{3 dice no 6}) &= \left(\frac{5}{6}\right)^3 \\
    &= 0.5787 \\
\end{align*}

Now we find how it fails after 6 throws.
\begin{align*}
    P(\text{No 6 in 6 rolls}) &= P(\text{3 dice no 6})^6 \\
    &= 0.0376 \\
\end{align*}

\begin{align*}
    &= \frac{P(X>10)}{P(\text{No 6 in 6 throws})} \\
    &= \frac{P(X\geq11)}{0.0376} \\
    &= \frac{(1-p)^{11-1}}{0.0376} \\
    &= \frac{\left(1-\frac{19}{27}\right)^{11-1}}{0.0376} \\
    &= 0.0001 \\
\end{align*}

Thus the probability of taking more than 10 throws given that the firt 6 throws
did not produce a single 6 is 0.0001 (to 4 decimal places).

\section{Fish}
This problem can be modelled with a hypergeometric distribution. $N$ is the
total number of fish, $r$ is the number of desired fish and $n$ are then number
of fish we catch.
\begin{align*}
    P(X=k) &= \frac{{}^rC_k \times {}^{N-r}C_{n-k}}{{}^NC_n} \\
    &= \frac{{}^rC_k \times {}^{21-r}C_{n-k}}{{}^{21}C_n}
    && \text{We leave $n$ and $r$ for each specific question} \\
\end{align*}

\subsection{Remaining gold fish}
Instead of looking at it like the cat chose 17 fish, we instead "choose" those
4 lucky fish that were spared. We will use the 15 goldfish as our focus.
Our PMF is now,
\begin{align*}
    P(X=k) &= \frac{{}^rC_k \times {}^{21-r}C_{n-k}}{{}^{21}C_n} \\
    &= \frac{{}^{15}C_k \times {}^{21-15}C_{4-k}}{{}^{21}C_4} \\
    &= \frac{{}^{15}C_k \times {}^{6}C_{4-k}}{5985} \\
\end{align*}

From basic rules of the choice function, we know neither of the numbers in a
choice can be negative and the top number must be larger (can't choose 5 items
from a set of 4). From these we know, $0 \leq k \leq 4$ this gives us the
domain. From here we evaluate every possible $k$ and determine the range,
all taken to 4 decimal places.
\begin{align*}
    P(X=0) &= 0.0025 \\
    P(X=1) &= 0.0501 \\
    P(X=2) &= 0.2632 \\
    P(X=3) &= 0.4561 \\
    P(X=4) &= 0.2281 \\
\end{align*}

The range is $[0.0025, 0.4561]$.

\subsection{Expected silver carp}
For this question we will now focus on the 6 silver carps and the number of
fish caught, $r=6$ and $n=17$. Luckily we have a formula for expectation,
\begin{align*}
    E(X) &= n\frac{r}{N} \\
    &= 17 \frac{6}{21} \\
    &= 4.8571
\end{align*}

We can expect the cat to catch 4.2857 of the 6 silver carps.

\subsection{Probability of more silver carps}
Let the number of goldfish the cat caught be $x$ and the number of caught
silver carps is $y$. Assume the cat did catch more silver carps, $x < y$.
The number of fish left in the pond is $6-y$ carps and $15-x$ goldfish,
it is completely reasonable to conclude there can't be a negative number of
fish left in the pond.
Also the number of fish caught must be 17, $x+y=17$.
Now we perform basic algebra,
\begin{align*}
    x &< y && \text{add $y$ to both sides} \\
    x+y &< 2y \\
    17 &< 2y \\
    8.5 &< y \\
    9 &\leq y && \text{Swap them} \\
    -y &\leq -9 && \text{Add 6 to both sides} \\
    6-y &\leq 6-9 \\
    6-y &\leq -3 && \text{$6-y$ is the silver carp left} \\
    \text{Num. of silver carp left} &\leq -3
\end{align*}

But now the number of silver carp left in the pond is negative which is
directly against our previous declaration for a positive number of fish. We
have arrived at a contradiction, the cat cannot catch more silver carp than
goldfish.

\section{Camera chips}
Both chips use a normal distribution as models for their lifetimes.
Since this is a normal distribution we can standardise our distributions
and compare them.
We standardise both to get their comparable Z scores for $X\geq24000$.
\begin{align*}
    Z &= \frac{x - \mu}{\sigma} \\
    Z &= \frac{24000 - 21000}{3000} && \text{Chip 1} \\
    Z &= 1 \\
    Z &= \frac{24000 - 22000}{1000} && \text{Chip 2} \\
    Z &= 2 \\
\end{align*}

Using a standard deviation table we find the following, (we need to reverse
these since they are $\leq$)
\begin{align*}
    P(Z \geq 1) &= 1 - .8643 && \text{Chip 1} \\
    &= 0.1587 \\
    P(Z \geq 2) &= 1 - .9772 && \text{Chip 2} \\
    &= 0.0228 \\
\end{align*}

We can make this comparison now because both distribution are in standard form,
they both follow the same PMF. Without out even looking at the actual
probabilities we have an intuitive understanding that Chip 1 will do better.
In Chip 2's case, achieving a 240000 hour
lifetime is given a Z score much higher than Chip 1's. Higher Z score means
less likely to achieve. Looking at the actual probabilities now, Chip 1 has a
probability of 0.1587 compared to Chip 2's 0.0228 (much lower). Therefore,
Chip 1 will have a higher probability of reaching the 24000 mark.

\section{Malfunction time}
We use the exponential distribution for the chance of malfunction.
The PDF ($f$) and CDF ($F$) are,
\begin{align*}
    f(x) &= \lambda e^{-\lambda x} && x\geq0 \\
    F(x) &= 1-e^{-\lambda x} && x\geq0 \\
\end{align*}

We expect a malfunction in 12 days. Luckily we have a simple formula for the
expectation of a exponential function.
\begin{align*}
    E(X) &= \frac{1}{\lambda} \\
    12 &= \frac{1}{\lambda} \\
    \lambda &= \frac{1}{12} \\
\end{align*}

So our complete model is now,
\begin{align*}
    f(x) &= \frac{1}{12} e^{-\frac{1}{12} x} && x\geq0 \\
    F(x) &= 1-e^{-\frac{1}{12} x} && x\geq0 \\
\end{align*}

\subsection{First day malfunction}
We're looking for $P(X \leq 1) = F(1)$.
\begin{align*}
    F(1) &= 1 - e^{-\frac{1}{12} 1} \\
    &= 0.0800 && \text{To 4 decimal places} \\
\end{align*}

The chance of a malfunction on the first day is 0.0800.

\subsection{Another 12 days}
We denote another feature of our model,
\begin{align*}
    P(X > x) &= 1 - P(X \leq x) \\
    &= 1 - (1 - e^{-\lambda x}) \\
    &= e^{-\lambda x} \\
\end{align*}

Now we can properly evaluate our problem,
\begin{align*}
    P(X > 24|X > 12) &= \frac{P(X>24 \cap X>12)}{P(X > 12)} \\
    &= \frac{P(X>24)}{P(X > 12)} \\
    &= \frac{e^{-\frac{1}{12} 24}}{e^{-\frac{1}{12} 12}} \\
    &= e^{-\frac{1}{12} 12} && \text{Note: memoryless} \\
    &= e^{-1} \\
    &= 0.3679 && \text{To 4 decimal places} \\
\end{align*}

Here we see the memoryless property pop up. It is trivial to
generalise this example but we will just move on. The probability of
malfunctioning after 12 days is 0.3679.

\section{Derive Z}
We know $Y \sim N(1,9)$ so we can represent it as an affine transformation
of the standard Gaussian random variable, $G$.
\begin{align*}
    Y &= \mu + \sigma G \\
    Y &= 1 + 3G
\end{align*}

Now we look at $Z$ as an affine transformation of $Y$ then $G$.
\begin{align*}
    Z &= 2Y - 5 \\
    Z &= 2(1 + 3G) - 5 \\
    Z &= 2 + 6G - 5 \\
    Z &= 6G - 3 \\
    Z &\sim N(-3, 36)
\end{align*}

Our random variable will be positive if $G > \frac{1}{2}$. Luckily we can
find the probability of this since $G$ is the normal Gaussian random variable.
Looking through the standard normal distribution table we find the following,
\begin{align*}
    P(G > \frac{1}{2}) &= 1 - P(G \leq \frac{1}{2}) \\
    &= 1 - 0.6915 \\
    &= 0.3085 \\
\end{align*}

The probability of having a positive $Z$ is 0.3085.

We now consider $Y \sim U(1,9)$. We again represent our random variable as
an affine tranformation of a more standard, $U(0,1)$.
\begin{align*}
    Y &= 1 + (9-1)\times U(0,1) \\
    Y &= 1 + 8\times U(0,1) \\
\end{align*}

We look at $Z$ as an affine transformation of $U(0,1)$ then the
equivalent distribution in standard notation.
\begin{align*}
    Z &= 2Y - 5 \\
    Z &= 2(1 + 8\times U(0,1)) - 5 \\
    Z &= 2 + 16U(0,1) - 5 \\
    Z &= 16U(0,1) - 3 \\
    Z &\sim U(-3, 13) \\
\end{align*}

In this case to get a positive $Z$ we need $U(0,1) > \frac{3}{16}$.
Thankfully it is quite easy to calculate the CDF for a uniform random number.

The height of $U(0,1)$ is also its PDF which is determined by,
\begin{align*}
    f(x) &= \frac{1}{b-a} && a < x < b \\
    f(x) &= \frac{1}{1-0} && 0 < x < 1 \\
    f(x) &= 1 && 0 < x < 1 \\
\end{align*}

Now we find $P\left(U(0,1) > \frac{3}{16}\right)
= 1 - P\left(U(0,1) < \frac{3}{16}\right)$.
\begin{align*}
    P(U(0,1) < x) = F(x) &= \frac{x}{1} && 0 < x < 1 \\
    P\left(U(0,1) < \frac{3}{16}\right) &= \frac{3}{16} \\
    &= 0.1875 \\
\end{align*}

Thus $P(U(0,1) > \frac{3}{16}) = 1 - 0.1875 = 0.8125$

\section{Countable discontinuities}
To prove a countable number of points $a$ we prove there exists an injective
function from those points $a$ to the rational numbers. Injective functions
prove that a domain is less than or equal to the range. Consider the function
the following function.

For every point $a$ a discontinuity exists from $F(a-)$ to $F(a)$. Since
$F(a-) \neq F(a)$ then we have an interval $(F(a-),F(a)]$. Since this is an
interval on the real numbers, there exists a rational number in there. Our
function will map $a$ to one of the rational numbers in $(F(a-),F(a)]$.

Now we just need to prove uniqueness and we've created an injective function.
All CDFs are monotone increasing thus
$(F(a_1-),F(a_1)] \cap (F(a_2-),F(a_2)] = \emptyset$. It is not possible for
any $a$s to share a rational number. Thus we have successfully constructed an
injective function from $a$ to the rationals so,
\begin{align*}
    \text{Num. of discontinuities} \leq \mathbb{Q}
\end{align*}

Since the rationals are countably infinite, the number of $a$ must be at least
countable.

\end{document}
