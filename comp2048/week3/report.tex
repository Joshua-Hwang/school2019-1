\documentclass{article}
\usepackage{amsmath}
\usepackage{amsfonts}
\usepackage{graphicx}
\usepackage{float}
\usepackage{verbatim}

\usepackage[margin=0.5in]{geometry}

\linespread{1.3}
\setlength{\parindent}{0em}
\setlength{\parskip}{1em}

\title{Assignment 1}
\author{Joshua Hwang (44302650)}
\date{11 March}

\begin{document}
\maketitle
\section{Caesar cipher}
\subsection{\texttt{test\_caesar.py}}
\verbatiminput{test_caesar.py}

The output is,
\begin{verbatim}
Message: The quick brown fox jumped over the lazy dog
Cypher Dict: {'a': 'b', 'b': 'c', 'c': 'd', 'd': 'e' ...
Encrypted Message: Uif rvjdl cspxo gpy kvnqfe pwfs uif mbaz eph
Decrypted Message: The quick brown fox jumped over the lazy dog
\end{verbatim}

\subsection{\texttt{test\_caesar\_break.py} part b and c}
\verbatiminput{test_caesar_break.py}

The output is,
\begin{verbatim}
Z : 1
y : 15
p : 27
  : 50
c : 14
x : 6
m : 2
e : 18
z : 18
w : 4
v : 5
f : 9
a : 6
l : 12
s : 7
d : 6
o : 7
h : 7
j : 6
q : 4
E : 2
g : 4
r : 5
t : 13
H : 1
n : 1
D : 1
S : 1
Max Ocurring Letter: p
Predicted Shift: 11
Message: One remember to look up at the stars and not down at your feet
Two never give up work
Work gives you meaning and purpose and life is empty without it
Three if you are lucky enough to find love remember it is there and dont throw it away
Stephen Hawking
\end{verbatim}

\section{Enigma cipher}
\subsection{\texttt{test\_enigma\_simple.py} part a and b}
\verbatiminput{test_enigma_simple.py}

The output is,
\begin{verbatim}
Message: Hello World
Encoded Message: Ncsmm Ywdpy
Decoded Message: Hello World
Decoded Message: What will I do? The same thing I do every night. Try and take over the world!
\end{verbatim}

\subsection{\texttt{test\_enigma\_break.py} part c, d and e}
\verbatiminput{test_enigma_break.py}

The output is,
\begin{verbatim}
Gdwm Qopjmw!
To all my loyal Spider Monkeys!
Half of you are to attack the enemy on their left flank.
Half of you are to attack the enemy on their right flank
and the rest of you shall come with me down the middle! Hail Shakes!

Tries: 11771
\end{verbatim}

My computer took roughly 5 seconds to crack the cipher.
We can use Moore's law as an approximation. Since 1940 was 80 years ago
there would've been 40 doublings since then, the time would have halved
every two years.
\begin{align*}
    \frac{5 \times 2^40}{60\times60\times24\times365}
    &\approx 174326
\end{align*}

A 1940s computer will take 174326 years to solve.

By adding two new rotors we now have ${}^5P_3 = 60$ possible rotor combos. Each
rotor can be set
Not only that but we a plug board where each plug can be placed in
any remaining letter giving us ${}^{26}C_10 = 5.3 \times 10^6$ (Note: you could
choose not using a plug and certain ordering are not equivalent
but we'll ignore those).
Multiplying these possibilities gives us $3.19 \times 10^8$ times
longer than our previous method.

Thus it would take 50 years to complete.

\section{Code Breaking}
Though we have no idea what encryption is being used we shall try some
rudimentary tactics. First we'll try a frequency check

The first part appears to be the work "ATTACK" where A is 19, T is 17 etc.
The next words are "PEARL" and "HARBOR" since they fit with the other A values
and also because of what happened in 1941. The remaining parts are "DECEMBER"
and "SEVEN" from context and the placement of other Es.

The full message read: "ATTACK PEARL HARBOR DECEMBER SEVEN".

\end{document}
