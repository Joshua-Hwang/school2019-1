\documentclass{article}
\usepackage{amsmath}
\usepackage{amsfonts}
\usepackage{graphicx}
\usepackage{float}

\linespread{1.3}
\setlength{\parindent}{0em}
\setlength{\parskip}{1em}

\newcommand{\diff}{\mathop{}\!\mathrm{d}}

\title{Assignment 4}
\author{Joshua Hwang (44302650)}
\date{26 May}

\begin{document}
\maketitle

\section{Prove statement}
Since $f$ is a continuous function we know $F(x) = \int_0^x f(t) \diff t$
exists and $F'(x) = f(x)$. Thus,
\begin{align*}
    \int_0^x \int_0^t f(u) \diff u \diff t
    &= \int_0^x F(t) \diff t \\
    &= \int_0^x 1 \times F(t) \diff t \\
    &= t \times F(t) \Big|_0^x - \int_0^x tf(t) \diff t
    && \text{Split into parts} \\
    &= x \times F(x) - \int_0^x tf(t) \diff t \\
    &= x \times \int_0^x f(t) \diff t - \int_0^x tf(t) \diff t
    && \text{From the statement we made above} \\
    &= \int_0^x xf(t) \diff t - \int_0^x tf(t) \diff t \\
    &= \int_0^x f(t)(x-t) \diff t \\
    &= \int_0^x f(u)(x-u) \diff u \\
    &= RHS \\
\end{align*}

\section{Integrate $\sin x$}
\subsection{Evaluate limits of sequences}
We evaluate each integral for the general $a_n$ and $b_n$.
\begin{align*}
    a_n &= \int_{-\pi n}^{\pi n} \sin x \diff x \\
    &= -\cos x \Big|_{-\pi n}^{\pi n} \\
    &= \cos (-\pi n) - \cos (\pi n) \\
    &= \cos (\pi n) - \cos (\pi n)
    && \text{Cosine is an even function} \\
    &= 0
\end{align*}

Since $a_n = 0$ $\forall n$, $\lim_{n \to \infty} a_n = 0$.

\begin{align*}
    b_n &= \int_{-2\pi n}^{\pi + 2\pi n} \sin x \diff x \\
    &= -\cos x \Big|_{-2\pi n}^{\pi + 2\pi n} \\
    &= \cos (-2\pi n) - \cos (\pi + 2\pi n) \\
    &= \cos (2\pi n) - \cos (\pi + 2\pi n)
    && \text{Cosine is an even function} \\
    &= \cos (0) - \cos (\pi)
    && \text{Cosine has period $2\pi$} \\
    &= 1 - -1 = 2
\end{align*}

Since $b_n = 2$ $\forall n$, $\lim_{n \to \infty} b_n = 2$.

\subsection{Improper integral}
The integral exists if the right hand side can be evaluated.
\begin{align*}
    \int_{-\infty}^{\infty} \sin x \diff x
    &= \lim_{b \to -\infty} \int_{b}^{0} \sin x \diff x
    + \lim_{a \to \infty} \int_{0}^{a} \sin x \diff x \\
    &= \lim_{b \to -\infty} \cos (b) - \cos (0)
    + \lim_{a \to \infty} \cos (0) - \cos (a) \\
    &= \lim_{b \to -\infty} \cos (b) - \lim_{a \to \infty} \cos (a) \\
\end{align*}

The right hand side limits does not exist hence our improper integral
does not exist.

\section{Prove series converges}
A series converges absolutely if the absolute version of the terms also
converges.
\begin{align*}
    \sum_{n=1}^\infty \left| \frac{n^2 + \cos n}{e^{n^3}} \right|
    &= \sum_{n=1}^\infty \frac{\left| n^2 + \cos n \right|}{e^{n^3}} \\
    &< \sum_{n=1}^\infty \frac{\left| n^2 \right| + \left| \cos n \right|}
    {e^{n^3}} \\
    &= \sum_{n=1}^\infty \frac{\left| n^2 \right|}{e^{n^3}}
    + \frac{\left| \cos n \right|}{e^{n^3}} \\
    &= \sum_{n=1}^\infty \frac{\left| n^2 \right|}{e^{n^3}}
    + \sum_{n=1}^\infty \frac{\left| \cos n \right|}{e^{n^3}} \\
\end{align*}

Now we just have to prove these two sums converge to prove our initial sum
converges.

We perform the ratio test on this series,
\begin{align*}
    \lim_{n \to \infty} \left| \frac{(n+1)^2}{e^{(n+1)^3}} \middle/ \frac{n^2}{e^{n^3}} \right|
    &= \lim_{n \to \infty} \left| \frac{(n+1)^2}{n^2} \frac{e^{n^3}}{e^{(n+1)^3}} \right| \\
    &= \lim_{n \to \infty} \left| \frac{(n+1)^2}{n^2} e^{n^3-(n+1)^3} \right| \\
    &= \left| \lim_{n \to \infty} \frac{(n+1)^2}{n^2} \lim_{n \to \infty} e^{-(3n^2+3n+1)} \right|
    && \text{This works because, as we shall show, the limits exist for both parts} \\
    &= \left| \lim_{n \to \infty} \frac{n^2 + 2n + 1}{n^2} \lim_{n \to \infty} e^{-(3n^2+3n+1)} \right| \\
    &= \left| \lim_{n \to \infty} \left(1 + \frac{2}{n} + \frac{1}{n^2} \right) \lim_{n \to \infty} e^{-(3n^2+3n+1)} \right| \\
    &= \left| 1 \times \lim_{n \to \infty} e^{-(3n^2+3n+1)} \right| \\
    &= \left| 1 \times \lim_{n \to \infty} 0 \right| \\
    &= 0 < 1
\end{align*}

Thus our ratio test shows that $\sum_{n=1}^\infty \frac{\left| n^2 \right|}{e^{n^3}}$
is convergent.

\section{The final question}

ZZZ - use the L definition of the integral

\end{document}
