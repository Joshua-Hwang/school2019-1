\documentclass{article}
\usepackage{amsmath}
\usepackage{amsfonts}
\usepackage{graphicx}
\usepackage{float}

\linespread{1.3}
\setlength{\parindent}{0em}
\setlength{\parskip}{1em}

\newcommand{\diff}{\mathop{}\!\mathrm{d}}

\title{Assignment 4}
\author{Joshua Hwang (44302650)}
\date{26 May}

\begin{document}
\maketitle

\section{Prove statement}
Since $f$ is a continuous function we know $F(x) = \int_0^x f(t) \diff t$
exists and $F'(x) = f(x)$. Thus,
\begin{align*}
    \int_0^x \int_0^t f(u) \diff u \diff t
    &= \int_0^x F(t) \diff t \\
    &= \int_0^x 1 \times F(t) \diff t \\
    &= t \times F(t) \Big|_0^x - \int_0^x tf(t) \diff t
    && \text{Split into parts} \\
    &= x \times F(x) - \int_0^x tf(t) \diff t \\
    &= x \times \int_0^x f(t) \diff t - \int_0^x tf(t) \diff t
    && \text{From the statement we made above} \\
    &= \int_0^x xf(t) \diff t - \int_0^x tf(t) \diff t \\
    &= \int_0^x f(t)(x-t) \diff t \\
    &= \int_0^x f(u)(x-u) \diff u \\
    &= RHS \\
\end{align*}

\section{Integrate $\sin x$}
\subsection{Evaluate limits of sequences}
We evaluate each integral for the general $a_n$ and $b_n$.
\begin{align*}
    a_n &= \int_{-\pi n}^{\pi n} \sin x \diff x \\
    &= -\cos x \Big|_{-\pi n}^{\pi n} \\
    &= \cos (-\pi n) - \cos (\pi n) \\
    &= \cos (\pi n) - \cos (\pi n)
    && \text{Cosine is an even function} \\
    &= 0
\end{align*}

Since $a_n = 0$ $\forall n$, $\lim_{n \to \infty} a_n = 0$.

\begin{align*}
    b_n &= \int_{-2\pi n}^{\pi + 2\pi n} \sin x \diff x \\
    &= -\cos x \Big|_{-2\pi n}^{\pi + 2\pi n} \\
    &= \cos (-2\pi n) - \cos (\pi + 2\pi n) \\
    &= \cos (2\pi n) - \cos (\pi + 2\pi n)
    && \text{Cosine is an even function} \\
    &= \cos (0) - \cos (\pi)
    && \text{Cosine has period $2\pi$} \\
    &= 1 - -1 = 2
\end{align*}

Since $b_n = 2$ $\forall n$, $\lim_{n \to \infty} b_n = 2$.

\subsection{Improper integral}
The integral exists if the right hand side can be evaluated.
\begin{align*}
    \int_{-\infty}^{\infty} \sin x \diff x
    &= \lim_{b \to -\infty} \int_{b}^{0} \sin x \diff x
    + \lim_{a \to \infty} \int_{0}^{a} \sin x \diff x \\
    &= \lim_{b \to -\infty} \cos (b) - \cos (0)
    + \lim_{a \to \infty} \cos (0) - \cos (a) \\
    &= \lim_{b \to -\infty} \cos (b) - \lim_{a \to \infty} \cos (a) \\
\end{align*}

The right hand side limits does not exist hence our improper integral
does not exist.

\section{Prove series converges}
A series converges absolutely if the absolute version of the terms also
converges.
\begin{align*}
    \sum_{n=1}^\infty \left| \frac{n^2 + \cos n}{e^{n^3}} \right|
    &= \sum_{n=1}^\infty \frac{\left| n^2 + \cos n \right|}{e^{n^3}} \\
    &< \sum_{n=1}^\infty \frac{\left| n^2 \right| + \left| \cos n \right|}
    {e^{n^3}} \\
    &= \sum_{n=1}^\infty \frac{\left| n^2 \right|}{e^{n^3}}
    + \frac{\left| \cos n \right|}{e^{n^3}} \\
    &= \sum_{n=1}^\infty \frac{\left| n^2 \right|}{e^{n^3}}
    + \sum_{n=1}^\infty \frac{\left| \cos n \right|}{e^{n^3}} \\
    &< \sum_{n=1}^\infty \frac{\left| n^2 \right|}{e^{n^3}}
    + \sum_{n=1}^\infty \frac{1}{e^{n^3}} \\
\end{align*}

Now we just have to prove these two sums converge to prove our initial sum
converges.

We perform the ratio test on this series,
\begin{align*}
    \lim_{n \to \infty} \left| \frac{(n+1)^2}{e^{(n+1)^3}} \middle/ \frac{n^2}{e^{n^3}} \right|
    &= \lim_{n \to \infty} \left| \frac{(n+1)^2}{n^2} \frac{e^{n^3}}{e^{(n+1)^3}} \right| \\
    &= \lim_{n \to \infty} \left| \frac{(n+1)^2}{n^2} e^{n^3-(n+1)^3} \right| \\
    &= \left| \lim_{n \to \infty} \frac{(n+1)^2}{n^2} \lim_{n \to \infty} e^{-(3n^2+3n+1)} \right|
    && \text{This works because, as we shall show, the limits exist for both parts} \\
    &= \left| \lim_{n \to \infty} \frac{n^2 + 2n + 1}{n^2} \lim_{n \to \infty} e^{-(3n^2+3n+1)} \right| \\
    &= \left| \lim_{n \to \infty} \left(1 + \frac{2}{n} + \frac{1}{n^2} \right) \lim_{n \to \infty} e^{-(3n^2+3n+1)} \right| \\
    &= \left| 1 \times \lim_{n \to \infty} e^{-(3n^2+3n+1)} \right| \\
    &= \left| 1 \times \lim_{n \to \infty} 0 \right| \\
    &= 0 < 1
\end{align*}

Thus our ratio test shows that $\sum_{n=1}^\infty \frac{\left| n^2 \right|}{e^{n^3}}$
is convergent.

We perform the ratio test on this series,
\begin{align*}
    \lim_{n \to \infty} \left| \frac{1}{e^{(n+1)^3}} \middle/ \frac{1}{e^{n^3}} \right|
    &= \lim_{n \to \infty} \left| \frac{e^{n^3}}{e^{(n+1)^3}} \right| \\
    &= \lim_{n \to \infty} \left| e^{n^3 - (n+1)^3} \right| \\
    &= \lim_{n \to \infty} \left| e^{-(3n^2 + 3n + 1)} \right| \\
    &= 0 < 1 \\
\end{align*}

Thus our ratio test shows that $\sum_{n=1}^\infty \sum_{n=1}^\infty \frac{1}{e^{n^3}}$
is convergent.

Since the right hand side converges the left hand side converges and
$\sum_{n=1}^\infty \frac{n^2 + \cos n}{e^{n^3}}$ converges absolutely.

\section{Taylor series}
We first observe
\[
    \cosh x = \frac{e^x + e^{-x}}{2} = f(x)
\]

This definition will allow us to find the derivatives for this function.
\begin{align*}
    f(x) &= \frac{e^x + e^{-x}}{2} \\
    f'(x) &= \frac{e^x - e^{-x}}{2} \\
    f''(x) &= \frac{e^x + e^{-x}}{2} = f(x) \\
    f'''(x) &= \frac{e^x - e^{-x}}{2} = f'(x) \\
\end{align*}

We notice that our second derivative lands back to our original function.
Thus every even derivative will land us on $\cosh x$ and every odd derivative
will land us on $\sinh x = \frac{e^x - e^{-x}}{2}$. This will make our
Taylor series much easier. We will use $a=0$,

\begin{align*}
    \sum_{n=0}^\infty \frac{f^{(n)}(0)}{n!} (x-0)^n
    &= \sum_{n=0}^\infty \frac{f^{(2n)}(0)}{(2n)!} x^{2n}
    + \sum_{n=0}^\infty \frac{f^{(2n+1)}(0)}{(2n+1)!} x^{2n+1} \\
    &= \sum_{n=0}^\infty \frac{f(0)}{(2n)!} x^{2n}
    + \sum_{n=0}^\infty \frac{f'(0)}{(2n+1)!} x^{2n+1} \\
\end{align*}

Now all we need to do is evaluate $f(0)$ and $f'(0)$.
\begin{align*}
    f(0) &= \frac{e^0 + e^{-0}}{2} = 1
    f'(0) &= \frac{e^0 - e^{-0}}{2} = 0
\end{align*}

Thus one of our sums will disapear.
\begin{align*}
    &= \sum_{n=0}^\infty \frac{f(0)}{(2n)!} x^{2n}
    + \sum_{n=0}^\infty \frac{f'(0)}{(2n+1)!} x^{2n+1} \\
    &= \sum_{n=0}^\infty \frac{x^{2n}}{(2n)!} \\
\end{align*}

From here it will be trivial for us to evaluate $\cosh 1 = f(1)$.
\begin{align*}
    f(x) &= \sum_{n=0}^\infty \frac{x^{2n}}{(2n)!} \\
    f(1) &= \sum_{n=0}^\infty \frac{1}{(2n)!} \\
\end{align*}

Now all that remains is finding enough terms to ensure our answer is correct
to 6 decimal places.

We make use of the remainder term to get our approximation to at least
6 decimal places. To do this we find the first term of our series that is
less than $10^{-6}$ and this will be our remainder term.
\begin{align*}
    \text{All equations are done to 4 significant figures}
    n=0 &: \frac{1}{(0)!} = 1 \\
    n=1 &: \frac{1}{(2)!} = 0.5 \\
    n=2 &: \frac{1}{(4)!} = 0.04167 \\
    n=3 &: \frac{1}{(6)!} = 0.001389 \\
    n=4 &: \frac{1}{(8)!} = 2.480 \times 10^{-5} \\
    n=5 &: \frac{1}{(10)!} = 2.756 \times 10^{-7} \\
\end{align*}

Thus if we sum the following we will get an approximation of $\cosh 1$ up to
6 decimal places.
\[
    1 + 0.5 + 0.04167 + 0.001389 + 2.480 \times 10^{-5} + 2.756 \times 10^{-7} = 1.543080
\]

\section{Prove something}
We attempt to prove by definitions,
\begin{align*}
    f(x) &= f(-x) \\
    a_0 + \sum_{n=1}^\infty a_n x^n &= a_0 + \sum_{n=1}^\infty a_n (-x)^n \\
    \sum_{n=1}^\infty a_n x^n &= \sum_{n=1}^\infty a_n (-x)^n \\
    \sum_{n=1}^\infty a_{2n} x^{2n} + \sum_{n=1}^\infty a_{2n+1} x^{2n+1}
    &= \sum_{n=1}^\infty a_{2n} (-x)^{2n} + \sum_{n=1}^\infty a_{2n+1} (-x)^{2n+1} \\
    \sum_{n=1}^\infty a_{2n} x^{2n} + \sum_{n=1}^\infty a_{2n+1} x^{2n+1}
    &= \sum_{n=1}^\infty a_{2n} (-1)^{2n} x^{2n} + \sum_{n=1}^\infty a_{2n+1} (-1)^{2n+1} x^{2n+1} \\
    \sum_{n=1}^\infty a_{2n} x^{2n} + \sum_{n=1}^\infty a_{2n+1} x^{2n+1}
    &= \sum_{n=1}^\infty a_{2n} x^{2n} - \sum_{n=1}^\infty a_{2n+1} x^{2n+1} \\
    \sum_{n=1}^\infty a_{2n+1} x^{2n+1} &= - \sum_{n=1}^\infty a_{2n+1} x^{2n+1} \\
    2 \sum_{n=1}^\infty a_{2n+1} x^{2n+1} &= 0 \\
    \sum_{n=1}^\infty a_{2n+1} x^{2n+1} &= 0 \\
\end{align*}

Since our function must work $\forall x \in \mathbb{R}$, we are restricted to
choosing our $a_{2n+1}$ to work for all $x$. Since terms of different degrees
cannot be summed to a single term (polynomials are a vector space) we can
simplify our equation for each individual term.
\[
    a_{2n+1} x^{2n+1} = 0
\]

We must conclude that $a_{2n+1} = 0$ for all $n$.

\section{Prove the limit doesn't exist}
Suppose the limit and is $L$. Then for any $\epsilon$ we can find a
$\delta > d((x,y,z), (0,0,0))$ that will satisfy $\epsilon > |f(x) - L|$.
\begin{align*}
    \epsilon &> \left| \frac{x^2 y z}{x^8 + y^4 + z^2} - L \right| \\
    &= \left| \frac{0^2 y z}{0^8 + y^4 + z^2} - L \right| \text{Along $x=0$} \\
    &= \left| L \right| \\
\end{align*}

From here we must conclude that $L = 0$ (since we're assuming it exists).

Now we must show that all paths to $(0,0,0)$ will give the same result for $L$.
\begin{align*}
    \epsilon &> \left| \frac{x^2 y z}{x^8 + y^4 + z^2} - L \right| \\
    &= \left| \frac{x^2 x^2 x^4}{x^8 + x^8 + x^8} - L \right| \text{Along $y=x^2$ and $z=x^4$} \\
    &= \left| \frac{x^8}{3x^8} - L \right| \\
    &= \left| \frac{1}{3} - L \right| \\
\end{align*}

From here we must conclude that $L=\frac{1}{3}$ but previously we concluded
that $L = 0 \neq \frac{1}{3}$. Thus we have created a contradiction, the limit
$L$ does not exist.

\end{document}
