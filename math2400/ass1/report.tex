\documentclass{article}
\usepackage{amsmath}
\usepackage{amsfonts}
\usepackage{graphicx}
\usepackage{float}

\linespread{1.3}
\setlength{\parindent}{0em}
\setlength{\parskip}{1em}

\title{Assignment 1}
\author{Joshua Hwang (44302650)}
\date{1 March}

\begin{document}
\maketitle

\section{Proof using formal definition of a limit}
Show $\lim_{n \to \infty} \frac{1}{\sqrt{n}} = 0$.

Given some $\epsilon > 0$, let $M > \frac{1}{\epsilon^2}$
so $\frac{1}{\sqrt{M}} < \epsilon$.
Now, $\forall n \geq M$
we get $\frac{1}{\sqrt{n}} \leq \frac{1}{\sqrt{M}}$
\begin{align*}
    \left|x_n - 0\right| &= \left|\frac{1}{\sqrt{n}} - 0\right| \\
    &= \frac{1}{\sqrt{n}} \leq \frac{1}{\sqrt{M}} < \epsilon
\end{align*}

Therefore, $\lim_{n \to \infty} \frac{1}{\sqrt{n}} = 0$.

\section{Prove a sequence is a Cauchy sequence}
We first note that $\forall y_n \geq 0$ because $|x_m - x_k| \leq y_k$.

Since $\{y_n\}^\infty_{n=1}$ converges to $0$. We pick an $M$ such that
$y_k < \epsilon$ for all $k > M$ for any arbitrarily chosen $\epsilon > 0$.
There exists such a $y_k < \epsilon$ because the definition of convergence
(through the $\epsilon$ - $N$ definition).

Now choose any $k > M$ and maintain $m > k > M$,
\begin{align*}
    |x_m - x_k| &\leq y_k < \epsilon \\
    |x_m - x_k| &< \epsilon \\
\end{align*}

Which is statisfied $\forall m \geq \forall k > M$.
Thus conforming to the definition of a Cauchy sequence proving it is a
Cauchy sequence.

\section{Prove a recursive sequence converges}
We first prove the sequence is monotonic increasing. The first case are
easy, $1 < \sqrt{2}$. From then on we use induction, let $k$ and $k+1$
be the last known values it works for.
\begin{align*}
    x_k &< x_{k+1} \\
    x_k+1 &< x_{k+2} \\
    \sqrt{2}^{x_k} &< \sqrt{2}^{x_{k+1}} \\
    x_k \log\left(\sqrt{2}\right) &< x_{k+1} \log\left(\sqrt{2}\right) \\
    x_k &< x_{k+1}
\end{align*}

Now we know the sequence is constantly increasing so all we need is to show we
have an upper bound. In this case we'll show no term in the sequence will be
higher than $2$. We will use induction once again to prove this.

The base case is obvious
\begin{align*}
    \sqrt{2} &< 2 \\
    \sqrt{2}^1 &< \sqrt{2}^2 \\
    1 &< \frac{\sqrt{2}^2}{\sqrt{2}^1} \\
    1 &< \sqrt{2} \\
\end{align*}

Knowing that $\sqrt{2} > 1$ from using a calculator we can conclude the base
case holds.

Now assume the $k$-th term works, $x_k < 2$, and
prove that the $k+1$-th term also works.
\begin{align*}
    x_k &< 2 \\
    \sqrt{2}^{x_k} &< \sqrt{2}^2 \\
    x_{k+1} &< 2
\end{align*}

Which means we have an upper limit at $2$ which means the sequence converges.

\section{Show a series is convergent}
\begin{align*}
    \sum^\infty_{n=1} \frac{(-1)^n}{n}
\end{align*}

First note that the sequence $\frac{1}{n}$ is monotone decreasing since
$n \in \mathbb{N}$.
\begin{align*}
    n < n+1 \\
    \frac{1}{n+1} < \frac{1}{n} \\
\end{align*}

Split the sequence into addition and subtraction terms
since $\frac{1}{n}$ are monotone decreasing each subtraction-addition pair
is actually decreasing thus the series is decreasing as $n$ gets larger.
\begin{align*}
    \sum^\infty_{n=1} \frac{(-1)^n}{n}
    &= \left(-\frac{1}{1} + \frac{1}{2}\right)
    + \left(-\frac{1}{3} + \frac{1}{4}\right) ... \\
\end{align*}

But every addition-subtraction pair is increasing,
\begin{align*}
    \sum^\infty_{n=1} \frac{(-1)^n}{n}
    &= -\frac{1}{1} + \left(\frac{1}{2} - \frac{1}{3}\right)
    + \left(\frac{1}{4} - \frac{1}{5}\right) ... \\
\end{align*}

Since every bracketed term increases the sum we take a lower bound of
$-\frac{1}{1}$. We now have a lower
bound and a monotone decreasing series therefore it is convergent.

\section{Work with a sequence with a variable initial term}
The sequence in question starts at $x_1 = a$ and follows with
$x_{n+1} = x_n \left(x_n + \frac{1}{n}\right)$

\subsection{When the sequence is monotone increasing and bounded}
Let $\lim_{n \to \infty} x_n = x$. Thus,
\begin{align*}
    x_{n+1} &= x_n \left(x_n + \frac{1}{n}\right) \\
    \lim_{n \to \infty} x_{n+1}
    &= \lim_{n \to \infty} x_n \left(x_n + \frac{1}{n}\right) \\
    x &= x \left(x\right) \\
    0 &= x(x-1) \\
\end{align*}

Since our sequence begins positive and is monotone increasing we can rule out
$0$. Thus our $\lim_{n \to \infty} x_n = 1$.

\subsection{Find $a$ so the sequence is not monotone}
Through an example, let $a = 0.6$.
\begin{align*}
    x_1 &= 0.6 \\
    x_2 &= 0.66 \\
    x_3 &= 0.6556
\end{align*}

\subsection{Find $a$ so the sequence is unbounded}
Consider a new sequence $y_{n+1} = y_n^2$.
We observe $x_{n+1} = y_{n+1}$
\begin{align*}
    \frac{1}{n} &> 0 \\
    x_n + \frac{1}{n} &> x_n \\
    x_n\left(x_n + \frac{1}{n}\right) &> x_n^2
\end{align*}

so if we can prove $x_n^2$ is unbounded positively then so is 
$\{x_n\}_{n=1}^\infty$.

We know use a ratio test to show our $\{y_n\}_{n=1}^\infty$ is unbounded.
\begin{align*}
    \lim_{n \to \infty} \frac{y_{n+1}}{y_n}
    &= \lim_{n \to \infty} \frac{y_n^2}{y_n} \\
    &= \lim_{n \to \infty} y_n \\
    &= ((...((a^2)^2)^2...)) \\
\end{align*}

Now we prove that $((...((a^2)^2)^2...))$ can unbounded when $a > 1$.
Consider a sequence where each successive term is the previous squared
(i.e. $x_{k+1} = x_k^2$). Let $x_0 > 1$, so $x_0 = 1 + \epsilon_0$ where
$\epsilon_0 \in \mathbb{R}$. Assume the $k$-th term works,
$x_k = 1 + \epsilon_k > 1$. Let's check the $k+1$-th term works as well.
\begin{align*}
    x_k &= 1 + \epsilon_k \\
    x_k^2 &= (1 + \epsilon_k)^2 \\
    x_{k+1} &= 1 + 2\epsilon_k + \epsilon_k^2 \\
    x_{k+1} &> 1 + 2\epsilon_k \\
\end{align*}

Note that $\epsilon_{k+1} > 2\epsilon_k$. With each term the $\epsilon$ value
is at least doubled. Thus we can conclude that $x_0 > 1$ will produce an
unbound sequence if successively squared.

If $a > 1$ then the ratio is well above 1 and the sequence is unbounded.
Since $\exists a$ where $\{y_n\}_{n=1}^\infty$ is unbounded there also
$\exists a$ where $\{x_n\}_{n=1}^\infty$ is unbounded.

\section{Determine if convergent}
Since all the questions below have denominators that are monotonic increasing
we can see that they are well defined and don't have $\frac{1}{0}$ issues.

\subsection{$\sum^\infty_{n=1} n\sin\left(\frac{1}{n}\right)$}
We use the fact that a series a convergent series $\sum x_n$ has a convergent
sequence $x_n$ with limit $\lim_{n \to \infty} x_n = 0$.
Thus we check if $\lim_{n \to \infty} n\sin\left(\frac{1}{n}\right) = 0$.

First we prove a well known limit, $\lim_{x \to 0} \frac{\sin x}{x} = 1$.
We use L'Hopital's rule to evaluate this limit. L'Hopital's rule applies when
the limit evaluates to $\frac{0}{0}$ or $\frac{\infty}{\infty}$. In this 
particular case it's clear that the limit will reduce to the first case.
\begin{align*}
    \lim_{x \to 0} \frac{\sin x}{x} &= \frac{\sin(0)}{0} \\
    &= \frac{0}{0}
\end{align*}

L'Hopital's rule means we differentiate the numerator and denominator and
evaluate the limit on those two.
\begin{align*}
    \lim_{x \to 0} \frac{\sin x}{x} &= \lim_{x \to 0} \frac{\cos(x)}{1} \\
    &= \frac{\cos(0)}{1} \\
    &= \frac{1}{1} \\
\end{align*}

Therefore we have proven the original claim
$\lim_{x \to 0} \frac{\sin x}{x} = 1$.

From here we adjust the limit by letting $n = \frac{1}{x}$.
So the limit equation becomes
\begin{align*}
    \lim_{x \to 0} \frac{1}{x} &= \lim_{n \to \infty} n \\
\end{align*}

Replacing these in the $\lim_{x \to 0} \frac{\sin x}{x} = 1$ gives
\begin{align*}
    \lim_{x \to 0} \frac{\sin x}{x} &= 1 \\
    \lim_{n \to \infty} n\sin\left(\frac{1}{n}\right) &= 1 \\
\end{align*}

So this reveals the terms of the series converge to 1 NOT 0. Therefore, the
series will diverge.

\subsection{$\sum^\infty_{n=1} \frac{1}{n^n}$}
From the lecture notes we can remove the head of this infinite sum and still
keep the property of convergence if and only if the original series converged
(K-tail). We begin the new series at $n=2$ so
$\sum^\infty_{n=2} \frac{1}{n^n}$. Now we show a convergent
geometric series has larger terms than $\frac{1}{n^n}$ (Note that this series
is monotone increasing since all terms of the series are positive).

Consider the geometric series $\sum^\infty_{n=2} \frac{1}{2^n}$. Using
induction we prove $\frac{1}{2^n} \geq \frac{1}{n^n} \quad \forall n \geq 2$
which means that
$\sum^\infty_{n=2} \frac{1}{2^n} \geq \sum^\infty_{n=2} \frac{1}{n^n}$.
The base case starts with $\frac{1}{4} \geq \frac{1}{4}$. Now assume the
$k$-th terms hold and see if the $k+1$-th terms also apply.
\begin{align*}
    \frac{1}{2^{k+1}} &> \frac{1}{(k+1)^{k+1}} \\
    \frac{1}{2}^{k+1} &> \frac{1}{(k+1)}^{k+1} \\
    \frac{1}{2} &> \frac{1}{(k+1)} && \text{Only applies when $k>1$} \\
\end{align*}

Since our series has $\forall n \geq 2$ the inductive step is valid. Since a
geometric series converges if $r > 1$ and
$\sum^\infty_{n=2} \frac{1}{2^n} \geq \sum^\infty_{n=2} \frac{1}{n^n}$.
Now we know our original sequence has an upperbound and is monotone increasing
therefore it will converge.

\subsection{$\sum^\infty_{n=2} \frac{1}{n\ln n}$}
We will use an integral test on this series. We may use this test because
$\frac{1}{n\ln n}$ is a decreasing, nonnegative function. The integral test
states $\sum^\infty_{n=2} \frac{1}{n}$ converges if and only if
$\int^\infty_{2} \frac{1}{n} dn$ converges.
\begin{align*}
    \int^\infty_{2} \frac{1}{n} dn &= \ln(\ln(n))|^\infty_2 \\
    &= \lim_{M \to \infty} \ln(\ln(M)) - \ln(\ln(2)) \\
    &= \lim_{M \to \infty} \ln(\log_2(M)) \\
    &= \infty
\end{align*}

The integral diverges therefore the original series also diverges.

\end{document}
