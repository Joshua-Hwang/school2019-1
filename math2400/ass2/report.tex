\documentclass{article}
\usepackage{amsmath}
\usepackage{amsfonts}
\usepackage{graphicx}
\usepackage{float}

\linespread{1.3}
\setlength{\parindent}{0em}
\setlength{\parskip}{1em}

\title{Assignment 2}
\author{Joshua Hwang (44302650)}
\date{8 April}

\begin{document}
\maketitle
\section{Show the limit}
Given $f(x) = 2x^2 + 5$ show,
\begin{align*}
    \lim_{x \to 1} f(x) &= 7
\end{align*}

If the limit is 7 then for all $\epsilon > |f(x) - 7|$
there exists $\delta > 0$ such that $\delta > |x - 1|$.

\begin{align*}
    \epsilon &> |f(x) - 7| \\
    &= |2x^2 + 5 - 7| \\
    &= |2x^2 - 2| \\
    &= 2|x-1||x+1| \\
\end{align*}

Try $\delta_1 = 1$ so,
\begin{align*}
    \delta &> |x - 1| \\
    1 &> |x - 1| &> |x| - 1 && \text{reverse triangle inequality} \\
    2 &> |x| \\
\end{align*}

So 
\begin{align*}
    2|x-1||x+1| &\leq 2|x-1|(|x|+|1|) &&\text{triangle inequality} \\
    &< 2|x-1|(2+|1|) \\
    &< 6|x-1| \\
\end{align*}

Now we choose $\delta_2 = \frac{\epsilon}{6}$. Thus our final choice of delta
is $\delta = min\{1, \frac{\epsilon}{6}\}$. We now verify our results.
\begin{align*}
    |2x^2 + 5 - 7| &= 2|x-1||x+1| \\
    &< 2\delta|x+1| && \delta > |x-1| \\
    &= 2\delta|(x-1)+2| \\
    &< 2\delta|1+2| && 1 \geq \delta > |x-1| \\
    &< 6\delta \\
    &< \epsilon && \delta \leq \frac{\epsilon}{6}
\end{align*}

We have been able to contruct a delta for any given $\epsilon$. Therefore
the limit holds.

\section{Show absolute of function is continuous}
Knowing $f(x)$ is continuous we have for any $\epsilon > |f(x) - f(c)|$ there
exists a $\delta > |x - c|$. Using the reverse triange inequality we get
\begin{align*}
    ||f(x)| - |f(c)|| &\leq |f(x) - f(c)| < \epsilon \\
    ||f(x)| - |f(c)|| &< \epsilon \\
\end{align*}

Thus the $\delta$ that exists for $f(x)$ is equally valid for $F(x)$ thus it is
continuous.

\section{There exists an interval that is positive}
Since $f(x)$ is continuous at $a$ then by definition we have for all
$|f(x) - f(a)| < \epsilon$ there will exist a $\delta < |x-a|$. We choose
$\epsilon = f(a)$ so the range of $f(x)$ is now $(f(a) - f(a), f(a) + f(a))
= (0, 2f(a))$. Which is a positive range. For such an epsilon we know there
exists a $\delta$.

\section{Continuity of functions}
We know all polynomials are continuous and we know $\log x$ is continuous.
When we add two continuous functions we produce another continuous function.
Now we convert our problem to use IVT. Let $g(x) = f(x) - 3$ so if there exists
and $x$ where $g(x) = 0$ we will find $f(x) = 3$.

\begin{align*}
    g(x) &= 0 \\
    f(x) - 3 &= 0 \\
    f(x) &= 3 \\
\end{align*}

Using a calculator we try $g(0.5) = \frac{1}{64} - \log(2) - 3$ which is less
than 0. Now take $g(1.5) = \frac{729}{64} + \log\left(\frac{3}{2}\right) - 3$
which is greater than 0. Since $g(x)$ is a sum of continuous functions it is
continuous and the IVT can be applied. There is a point which is greater than 0
and a point less than 0 thus there exists a point where it is 0. Therefore,
there is a point, $x$, where $x^6 + \log(x) = 3$.

\section{Fixed point}
We use the same technique as before. Construct $g(x) = f(x) - x$ so when
$g(x) = 0 \to f(x) = x$. Since $f(x)$ is continuous, $g(x)$ is also continuous.

Try $g(0)$.
Since the domain and range of $f(x)$ is $[0,1]$, $f(x) \geq 0$.
If $f(0) = 0$ we're done thus our only interesting option is $f(0) > 0$.
Thus $g(0) \geq 0$.
Now try $g(1)$
Since the domain and range of $f(x)$ is $[0,1]$, $f(x) \leq 1$.
If $f(1) = 1$ we're done thus our only interesting option is $f(1) < 1$.
Thus $g(1) \leq 0$.

We now apply IVT to show there exists $g(x) = 0$ which means $f(x) = x$.

\section{Restricted function}
Counter example, $f(x) = \left|\frac{1}{0.5 + x}\right|$.
We show it is bounded on the integers but unbounded on the reals.

Since the function is absolute it is bounded below by 0. It is also monotone
increasing until it hits 0 then is monotone decreasing. We first try to show
the monotone increasing to 0. For clarity let $a = -x$
\begin{align*}
    f(x) &\leq f(x+1) && x < 0 \\
    f(-a) &\leq f(-a+1) && a > 0 \\
    \left|\frac{1}{0.5 - a}\right| &\leq \left|\frac{1}{0.5 - a + 1}\right| \\
    \left|\frac{1}{0.5 - a}\right| &\leq \left|\frac{1}{1.5 - a}\right| \\
    \frac{1}{\left|0.5 - a\right|} &\leq \frac{1}{\left|1.5 - a\right|} \\
    \left|0.5 - a\right| &\geq \left|1.5 - a\right| \\
    \left|a - 0.5\right| &\geq \left|a - 1.5\right| \\
\end{align*}

Since $a > 0$, the inequality holds. For $x < 0$ the function is monotone
increasing. A finite monotone increasing function will have it's upper bound
as its final answer. In this case it is $x=-1$ so $f(-1) = 2$.

The other side is monotone decreasing as we will demonstrate.
\begin{align*}
    f(x) &\geq f(x+1) && x \geq 0 \\
    \left|\frac{1}{0.5 + x}\right| &\geq \left|\frac{1}{0.5 + x + 1}\right| \\
    \frac{1}{\left|0.5 + x\right|} &\geq \frac{1}{\left|0.5 + x + 1\right|} \\
    \left|0.5 + x\right| &\leq \left|0.5 + x + 1\right| \\
    \left|0.5 + x\right| &\leq \left|1.5 + x\right| \\
\end{align*}

Since $x \geq 0$, the inequality holds. For $x \geq 0$ the function is monotone
decreasing. Recall the function is absolute so it is bounded below by 0 and
it's maximum value is the first possible answer. In this case it is $x=0$
so $f(0) = 2$. 

Thus we have a lower bound and upper bound that exist for this function
restricted to the integers. We now show such limits do not exist in the reals.

Assume there exists an upper bound $L$. Choose $x = \frac{1}{L + 1} - 0.5$ so
$f\left(\frac{1}{L+1} - 0.5\right) = L+1$ which is higher
than the upper bound.
Therefore there is no upper bound and it's unbounded over $\mathbb{R}$.

\end{document}
