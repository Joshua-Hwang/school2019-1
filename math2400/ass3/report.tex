\documentclass{article}
\usepackage{amsmath}
\usepackage{amsfonts}
\usepackage{graphicx}
\usepackage{float}

\linespread{1.3}
\setlength{\parindent}{0em}
\setlength{\parskip}{1em}

\title{Assignment 3}
\author{Joshua Hwang (44302650)}
\date{6 May}

\begin{document}
\maketitle

\section{Show that $f$ is differentiable at 0}
\begin{align*}
    f(x)
    &=
    \begin{cases}
        x^2 & \text{$x$ is rational} \\
        0 & \text{otherwise}
    \end{cases}
\end{align*}

\subsection{My first solution}
To show it is differentiable we must show the following exists,
\begin{align*}
    \lim_{x \to 0} \frac{f(x) - f(0)}{x - 0}
    &= \lim_{x \to 0} \frac{f(x)}{x}
\end{align*}

We consider both $f(x) = 0$ and $f(x) = x^2$ for different values of $x$.
First $f(x) = 0$.
\begin{align*}
    \lim_{x \to 0} \frac{f(x)}{x} &= \lim_{x \to 0} \frac{0}{x} \\
    &= 0 \\
\end{align*}

This is fine since ALL points $x$ close to 0 have the expression evaluate to
$\frac{0}{x} \to 0$. By definition of limit,
\begin{align*}
0 < |x - 0| < \delta \\
x \neq 0
\end{align*}

Now consider $f(x) = x^2$.
\begin{align*}
    \lim_{x \to 0} \frac{f(x)}{x} &= \lim_{x \to 0} \frac{x^2}{x} \\
    &= \lim_{x \to 0} x \\
    &=  0
\end{align*}

Both possiblities have been explored and evaluate to the same answer. Since
$f'(x)$ exists the function is differentiable.

\subsection{The "better" solution}
Another solution removes the need to explore rational and irrational $x$.
To show it is differentiable we must show the following exists,
\begin{align*}
    \lim_{x \to 0} \frac{f(x) - f(0)}{x - 0}
    &= \lim_{x \to 0} \frac{f(x)}{x}
\end{align*}

We note the following,
\begin{align*}
    0 &\leq f(x) &\leq x^2 \\
    0 &\leq \frac{f(x)}{x} &\leq x && \div x && \text{Where $x > 0$} \\
    0 &\geq \frac{f(x)}{x} &\geq x && \div x && \text{Where $x < 0$} \\
\end{align*}

Now we find if our expression has a result as $x \to 0$ from both sides.
We note that $\lim_{x \to 0} x = 0$. Thus through squeeze theorem we can
conclude $\lim_{x \to 0} \frac{f(x)}{x} = 0$. Since $\frac{f(x)}{x}$ exists at
0 we know $f$ is differentiable at 0.

\section{Show Lipschitz continuous iff $f'$ is bounded}
First suppose $f$ is Lipschitz continuous.
\begin{align*}
    |f(x) - f(y)| &\leq C|x - y|
\end{align*}

\begin{align*}
    |f'(x)| &= \lim_{x \to y} \frac{|f(x) - f(y)|}{|x - y|} \\
    &\leq \lim_{x \to y} \frac{C|x - y|}{|x - y|} \\
    &= C \\
\end{align*}

Thus our function is bounded.

Now suppose $f'$ is bounded by $L$, $|f'| \leq L$.
We may apply the Mean Value Thereom in this scenario since our function is
differentiable. By the Mean Value Thereom we have for all $a$ and $b$,
\begin{align*}
    f(b) - f(a) &= f'(c)(b - a) \\
    |f(b) - f(a)| &= |f'(c)||b - a| \\
    |f(b) - f(a)| &\leq L|b - a| \\
\end{align*}

Which is the defintion of Lipschitz continuous.

\section{Use lower and upper sums to show integrable}
First observe our function is monotone increasing.
Thus we produce the following expressions for the lower sum and upper sum,
\begin{align*}
    \sum_{i \in P \textbackslash \{0\}} f\left(\frac{i - 1}{n}\right) \times \frac{1}{n}
    &= \frac{1}{n^2} \sum_i^n i-1 && \text{Lower sum} \\
    &= \frac{1}{n^2} \frac{(n-1)n}{2} \\
    &= \frac{1}{2} - \frac{1}{2n} \\
    \sum_{i \in P \textbackslash \{0\}} f\left(\frac{i}{n}\right) \times \frac{1}{n}
    &= \frac{1}{n^2} \sum_i^n i && \text{Upper sum} \\
    &= \frac{1}{n^2} \frac{n(n+1)}{2} \\
    &= \frac{1}{2} + \frac{1}{2n} \\
\end{align*}

Now we prove it's integrable,
\begin{align*}
    U(P,f) - L(P,f) &< \epsilon \\
    \frac{1}{2} + \frac{1}{2n} - \left(\frac{1}{2} - \frac{1}{2n}\right)
    &< \epsilon \\
    \frac{2}{2n} &< \epsilon \\
    \frac{1}{n} &< \epsilon \\
\end{align*}

Thus choosing $n = \frac{2}{\epsilon}$ will give a partition that satisfies
the equation.

\section{Show mean value theorem of integrals}
By Fundamental Theorem of Calculus we may construct $F(x) = \int_a^x f$ and
$F'(x) = f(x)$. We now use the Mean Value Theorem since $F$ is differentiable.
\begin{align*}
    \frac{F(b) - F(a)}{b - a} &= F'(c) \\
    F(b) - F(a) &= F'(c)(b - a) \\
    F(b) - F(a) &= F'(c)(b - a) \\
    \int_a^b f &= f(c)(b - a) && \text{QED}
\end{align*}

\section{Show $f(x) = 0$}
Suppose $\int_a^b f = 0$ but $\exists c$ such that $f(c) > 0$,
(also note $f(x) \geq 0$). From the
previous assignment we've proven that if $f(c) > 0$ then there exists an
interval where the function is greater than 0, $(e, f)$.

From here we make use of the additivity property of the integral.
\begin{align*}
    \int_a^b f &= \int_a^e f + \int_e^f f + \int_f^b f \\
    &= 0 + \int_e^f f + 0 \\
    &= 0 + f(g)(f-e) + 0
    && \text{From the previous question we know $g$ exists} \\
    &> 0
\end{align*}

But we initially stated $\int_a^b f = 0$ which leads to a contradiction. We
cannot have any point $c$ such that $f(c) > 0$. Thus $f(x) = 0$.

\section{Show there exists $f(c) = g(c)$}
First we note $a \neq b$ since $[a,b]$ is an interval.
Let $H(x) = f - g$ from Q4 we have,
\begin{align*}
    \int_a^b f - g &= 0 \\
    \int_a^b H &= 0 \\
    H(c)(b - a) &= 0 && \text{There exists a $c$ where this holds} \\
    (f(c) - g(c))(b - a) &= 0 && \text{$b - a \neq 0$} \\
    f(c) - g(c) &= 0 \\
    f(c) &= g(c) \\
\end{align*}

\end{document}
