\documentclass{article}
\usepackage{mathtools}
\usepackage{amsmath}
\usepackage{amssymb}
\usepackage{amsfonts}
\usepackage{graphicx}
\usepackage{float}
\usepackage{verbatim}

\linespread{1.3}
\setlength{\parindent}{0em}
\setlength{\parskip}{1em}
\setcounter{secnumdepth}{0}
\setcounter{MaxMatrixCols}{20}

\title{Assignment 3}
\author{Joshua Hwang (44302650)}
\date{5 May}

\begin{document}
\maketitle

\section{Hill cipher}
\subsection{Find inverse}
We first show the standard inverse for $2\times2$ matrices works in modular
arithmetic. Consider $a$, $b$, $c$, $d$ to be equivalence classes in
$\mathbb{Z}_m^*$ (those with multiplicative inverses).
We prove this with the definition of an inverse matrix.
\begin{align*}
    (ad-bc)^{-1}
    \begin{bmatrix}
        d & -b \\
        -c & a
    \end{bmatrix}
    \begin{bmatrix}
        a & b \\
        c & d
    \end{bmatrix}
    &=
    (ad-bc)^{-1}
    \begin{bmatrix}
        ad-bc & 0 \\
        0 & ad-bc
    \end{bmatrix} \\
    &=
    \begin{bmatrix}
        1 & 0 \\
        0 & 1
    \end{bmatrix}
\end{align*}

The inverse is also unique. Consider the right inverse $Ar=I$ and the left
inverse $lA=I$.
\begin{equation}
    l = l(Ar) = r
\end{equation}

We attempt to find $K^{-1}$ through the standard form of the $2\times2$
matrix.
\begin{align*}
    (ad-bc)^{-1}
    \begin{bmatrix}
        d & -b \\
        -c & a
    \end{bmatrix}
    &=
    (15-8)^{-1}
    \begin{bmatrix}
        5 & -2 \\
        -4 & 3
    \end{bmatrix}
\end{align*}

It simply requires us to find $7^{-1}$ in $\mathbb{Z}_{26}$ (which exists since
they are coprime).

By using the extended Euler algorithm we can find this inverse.
\begin{align*}
    26 &= 7\times3 + 5 \\
    7 &= 5\times1 + 2 \\
    5 &= 2\times2 + 1 && \text{The GCF is 1} \\
    1 &= 5 - 2\times2 \\
    1 &= 5 - 2(7-5) \\
    1 &= 3\times5 - 2\times7 \\
    1 &= 3(26 - 7\times3) - 2\times7 \\
    1 &= 3\times26 - 7\times11 \\
\end{align*}

Thus $7^{-1} = -11 = 15$ in $\mathbb{Z}_{26}$. Our inverse matrix is thus,
\begin{align*}
    15
    \begin{bmatrix}
        5 & -2 \\
        -4 & 3
    \end{bmatrix}
    &=
    \begin{bmatrix}
        23 & 22 \\
        18 & 19
    \end{bmatrix}
\end{align*}

\subsection{Recover plaintext}
Using the matrix above and the vectors produced by this message,
\begin{verbatim}
WOPSBOEALHMU
\end{verbatim}
\begin{align*}
    \begin{pmatrix}
        22 \\ 14
    \end{pmatrix} \\
    \begin{pmatrix}
        15 \\ 18
    \end{pmatrix} \\
    \begin{pmatrix}
        1 \\ 14
    \end{pmatrix} \\
    \begin{pmatrix}
        4 \\ 0
    \end{pmatrix} \\
    \begin{pmatrix}
        11 \\ 7
    \end{pmatrix} \\
    \begin{pmatrix}
        12 \\ 20
    \end{pmatrix} \\
\end{align*}

We then decrypt using $K^{-1}x$ where $x$ will be each of the vectors above.
\begin{align*}
    \begin{bmatrix}
        23 & 22 \\
        18 & 19
    \end{bmatrix}
    \begin{pmatrix}
        22 \\ 14
    \end{pmatrix}
    &= 
    \begin{pmatrix}
        23\times22 + 22\times14 \\
        18\times22 + 19\times14 \\
    \end{pmatrix} \\
    &=
    \begin{pmatrix}
        8 \\
        12 \\
    \end{pmatrix} \\
\end{align*}

We repeat this process for each vector and get the following result.
\begin{verbatim}
imnotyourtoy
\end{verbatim}

\section{Unique ciphers}
\subsection{Shift cipher}
\begin{verbatim}
UJGCXIJGT
\end{verbatim}

I use a Python program to exhaust all possibilities.
\verbatiminput{q2a.py}

The most satisfactory result was \texttt{furniture} with shift 15.

\subsection{Affine cipher}
\begin{verbatim}
BWXHNHQCDXNWEFZTHJHBUNWWSRMHEWP
\end{verbatim}

Since we know \texttt{it} maps to \texttt{BW} in our message we can produce
a set of simulataneous equations. (\texttt{i=8, t=19, b=1, w=22})
\begin{align*}
    \begin{cases}
        8a + b &= 1 \bmod 26 \\
        19a + b &= 22 \bmod 26 \\
    \end{cases} \\
    \begin{cases}
        b &= 1 - 8a \bmod 26 \\
        b &= 22 - 19a \bmod 26 \\
    \end{cases} \\
    1 - 8a &= 22 - 19a \bmod 26 \\
    11a &= 21 \bmod 26 \\
    a &= 21(11^{-1}) \bmod 26 \\
\end{align*}

We find $11^{-1}$ in $\mathbb{Z}_{26}$. From the previous question we already
know,
\begin{align*}
    1 &= 3\times26 - 7\times11 \\
\end{align*}

So $a = 9 \bmod 26$. Now we must find $a^{-1}$.
\begin{align*}
    26 &= 2\times9 + 8 \\
    9 &= 8 + 1 && \text{The GCF is 1} \\
    1 &= 9 - 8 \\
    1 &= 9 - (26 - 2\times9) \\
    1 &= -26 + 3\times9 \\
\end{align*}

Thus $a^{-1} = 3$.

Now we find $b$.
\begin{align*}
    8a + b &= 1 \bmod 26 \\
    8(9) + b &= 1 \bmod 26 \\
    20 + b &= 1 \bmod 26 \\
    b &= 7 \bmod 26 \\
\end{align*}

With these we may now decode the text. Another Python progam was written to do
this job.
\begin{verbatim}
msg = "bwxhnhqcdxnwefzthjhbunwwsrmhewp"
ans=""
# iterate over each character in the message
for c in msg:
  i = ord(c) - ord('a')
  # use the decoding algorithm for affine ciphers
  i_d = (3*(i-7) + 26)%26
  c_d = chr(i_d + ord('a'))
  ans += c_d

print(ans)
\end{verbatim}

The end result is \texttt{itwasablowstruckagainsttheparty}.

\subsection{Vigenere cipher}
\begin{verbatim}
LAOGINXBVRYTHRMYAHGLAKHLIPJIVWAMSYCVKSFWWWOAHOXOEM
FZOYEJZSOYLMCALGESBJHTFZENBCYILLRBVATBTVSRKNWMLVRV
WWWAXGEOQCFTFOSJHITLKWFNAAGUESGFPLFGPWAKKXFIEFMGUM
KYCEIZXOQASLHUVGUPVRYPWGLETRQIFXRAIJOSFEFWVRJWEHJM
DWZLIPVWGIVUIGLALANRFXFNWZXPRRLHJRVZBGUSKMSFWKAOAH
OTGNWWTGLEFWUEEUXTHPSLSIIJISELSIGBRWGSIIJLSRQKLCZY
UAOGSFXGREKXOFAZXBBRWAOFXGIZNCSIOEXUXFGEAGZLRGHBRP
GHYVRYTHQSJBOAKJTMGLSMBVKZMQBYDWVNZWUSYMWOSQXZTHUI
ZTRCEKLSQXZKCHKZTHEEYXRLEKACEVAUZREKTBLXJTURHQHTBY
JTURXZHGRJAGSYCKAOCIVYWAKWKGPSMERAINXFUENXQYYLVVRH
SDBVJWYCEWAGBBVLACFIKFWYMFZZVTKAOIIUKWRHGNHBRYHRNR
VZCBHFXGFLWAWZWWETPSMERASLASYTOHBQIJBBTELMVRGSEABJ
ZBGQIEXOASMKOAHXHFNQGFSAXXXZGOWXBYCLASGIJKWOPWIZRE
KNFRSXTRBYTESYMXXWGASLOFQSEZCEJMMTSLNDEELASEMFTVHV
JRPLPSWMAEJUCESMZVJLGPOFENXFLGDXJRVOHANROBHUAZTHYS
JWVRRJRIFIVMCQIKVFVFWTGGLWKSZEAGGBJJXOYPQKSZEJDOOP
WNUYMFXGFWZXVNHHKCIIVTBRBUXZYIFMKVJWMCBRWHTBYJFCFX
LXRVSMLOZFSLGNHGKGNRVAOIMFZPHVAXRUIJAIFFSGRCVGISEP
QBBNQSKPYIETIFSDXIZAZBQUWZXVNHZXFFIDYRRWAZBRHSGRZE
JKWRHGYTUIJWOHKZMSEWLHGBQWKWPLJTHUIJXZQIJEMZIFLVRH
WOCGIVASEWWETASOMCGLWIZREKNFRWGYTEIFVVSMUMWBRXKSAG
ZVCBOWKM
\end{verbatim}

Kasiski's test was completed with a Python program. The comments of the program
should explain the process.
\verbatiminput{q2c_1.py}

If we were to use all distances, we would end up selecting information that
randomly coincided. Thus instead of looking for the greatest common factor
between all distances, we attempt to find the greatest (most) common factor.
We did this by slowing checking how many distances had a 3,4 then
5 as a common factor. This could be extended to 6,7,8, etc.
\begin{verbatim}
Factor:Frequency
3 : 4
4 : 3
5 : 14
\end{verbatim}

We also explore using quadgrams and quartgrams but we could
not conclude anything interesting from this, not enough n-grams available.

We now employ Friedman's first method.
Below we run through each $m < 5$. We get the $\sum{f_i(f_i-1)}$ from our
unique problem sheet.
\begin{align*}
    m &= 1 \\
    n &= 1058 \\
    I_1 &= \frac{\sum{f_i(f_i-1)}}{n(n-1)} = \frac{46488}{1118306} = 0.0416 \\
    m &= 2 \\
    n &= \frac{1058}{2} \\
    I_1 &= \frac{11380}{279312} = 0.0407 \\
    I_2 &= 0.0416 \\
    m &= 3 \\
    n &= \frac{1058}{3} \\
    I_1 &= 0.0424 \\
    I_2 &= 0.0411 \\
    I_3 &= 0.0405 \\
    m &= 4 \\
    n &= \frac{1058}{4} \\
    I_1 &= 0.0404 \\
    I_2 &= 0.0448 \\
    I_3 &= 0.0409 \\
    I_4 &= 0.0398 \\
    m &= 5 \\
    n &= \frac{1058}{4} \\
    I_1 &= 0.0404 \\
    I_2 &= 0.0448 \\
    I_3 &= 0.0409 \\
    I_4 &= 0.0398 \\
\end{align*}

For $m=5$ we get 0.065, 0.065, 0.064, 0.061 and 0.065. This
fits very well with the English language index of 0.065.

Friedman's second method is given by,
\begin{align*}
    m &\approx \frac{n(I_l - I_0)}{(n-1)I_T - nI_0 + I_l}
    &= \frac{1058(0.0667 - 0.0385)}{(1058-1)46488 - (1058)0.0385 + 0.0667}
    &= 9.1856
\end{align*}

This method seems to suggest a key length of 9. Of course we know
from external information this could not be possible (highest key length is 5).
Thus we must chalk this up to a failure in approximation, with more data
Friedman's second method would work much better.

\subsubsection{Decrypt}
From here we perform frequency analysis on each segment of the cipher text.
Those encoded with the first letter of the key, second letter, etc.
\verbatiminput{q2c_2.py}

Our resulting guess came out to be \texttt{ftone} which was not the actual key.
Analysing those decoded with the \texttt{tone} part of our key actually showed
common parts of words which lead us to the conclusion the \texttt{f} was
incorrectly guessed. From here we bruteforce each possible first letter to
arrive at \texttt{stone} which properly decoded the message.
\begin{verbatim}
thateveningateightthirtyexquisitelydressedandweari
ngalargebuttonholeofparmavioletsdoriangraywasusher
edintoladynarboroughsdrawingroombybowingservantshi
sforeheadwasthrobbingwithmaddenednervesandhefeltwi
ldlyexcitedbuthismannerashebentoverhishostessshand
wasaseasyandgracefulaseverperhapsoneneverseemssomu
chatoneseaseaswhenonehastoplayapartcertainlynoonel
ookingatdoriangraythatnightcouldhavebelievedthathe
hadpassedthroughatragedyashorribleasanytragedyofou
ragethosefinelyshapedfingerscouldneverhaveclutched
aknifeforsinnorthosesmilinglipshavecriedoutongodan
dgoodnesshehimselfcouldnothelpwonderingatthecalmof
hisdemeanourandforamomentfeltkeenlytheterribleplea
sureofadoublelifeitwasasmallpartygotupratherinahur
rybyladynarboroughwhowasaverycleverwomanwithwhatlo
rdhenryusedtodescribeastheremainsofreallyremarkabl
euglinessshehadprovedanexcellentwifetooneofourmost
tediousambassadorsandhavingburiedherhusbandproperl
yinamarblemausoleumwhichshehadherselfdesignedandma
rriedoffherdaughterstosomerichratherelderlymenshed
evotedherselfnowtothepleasuresoffrenchfictionfrenc
hcookery
\end{verbatim}

\section{RSA and Chinese Remainder Theorem}
We first let $s = x^3$ thus our equations can become,
\begin{align*}
    s &= 42710 \bmod 43807 \\
    s &= 15083 \bmod 48721 \\
    s &= 40156 \bmod 44767 \\
\end{align*}

From here we can perform the Chinese Remainder Theorem.
\begin{align*}
    s &= a_1 (m_2 \times m_3) y_1 \\
      &+ a_2 (m_1 \times m_3) y_2 \\
      &+ a_3 (m_1 \times m_2) y_3 \bmod (m_1 \times m_2 \times m_3) \\
    s &= 42710 (48721 \times 44767) y_1 \\
      &+ 15083 (43807 \times 44767) y_2 \\
      &+ 40156 (43807 \times 48721) y_3 \bmod (43807 \times 48721 \times 44767)
\end{align*}

Where $y_i$ is $(\prod_{n \neq i} m_n)^{-1} \bmod m_i$.

Since these are very large numbers a Python program was written to complete
this.
\verbatiminput{q3.py}

The $y$s were calculated with a Python program but have been copied below.
\begin{align*}
    m_1 &= 2181093007 \\
    m_2 &= 1961107969 \\
    m_3 &= 2134320847 \\
    m &= 95547141357649 \\
    \text{Finding inverse of 2181093007 mod 43807} \\
    1 &= 1 \times 1 + 22 \times 0 \\
    1 &= 22 \times -6 + 133 \times 1 \\
    1 &= 133 \times 19 + 421 \times -6 \\
    1 &= 421 \times -120 + 2659 \times 19 \\
    1 &= 2659 \times 619 + 13716 \times -120 \\
    1 &= 13716 \times -1358 + 30091 \times 619 \\
    1 &= 30091 \times 1977 + 43807 \times -1358 \\
    1 &= 43807 \times -98432234 + 2181093007 \times 1977 \\
    1 &= 2181093007 \times 1977 + 43807 \times -98432234 \\
    \text{Found inverse 1977} \\
    y_1 &= 1977 \\
    \text{Finding inverse of 1961107969 mod 48721} \\
    1 &= 1 \times 1 + 9 \times 0 \\
    1 &= 9 \times -1 + 10 \times 1 \\
    1 &= 10 \times 2 + 19 \times -1 \\
    1 &= 19 \times -3 + 29 \times 2 \\
    1 &= 29 \times 8 + 77 \times -3 \\
    1 &= 77 \times -11 + 106 \times 8 \\
    1 &= 106 \times 1009 + 9723 \times -11 \\
    1 &= 9723 \times -4047 + 38998 \times 1009 \\
    1 &= 38998 \times 5056 + 48721 \times -4047 \\
    1 &= 48721 \times -203513103 + 1961107969 \times 5056 \\
    1 &= 1961107969 \times 5056 + 48721 \times -203513103 \\
    \text{Found inverse 5056} \\
    y_2 &= 5056 \\
    \text{Finding inverse of 2134320847 mod 44767} \\
    1 &= 1 \times 1 + 6 \times 0 \\
    1 &= 6 \times -1 + 7 \times 1 \\
    1 &= 7 \times 3 + 20 \times -1 \\
    1 &= 20 \times -4 + 27 \times 3 \\
    1 &= 27 \times 7 + 47 \times -4 \\
    1 &= 47 \times -95 + 638 \times 7 \\
    1 &= 638 \times 102 + 685 \times -95 \\
    1 &= 685 \times -197 + 1323 \times 102 \\
    1 &= 1323 \times 299 + 2008 \times -197 \\
    1 &= 2008 \times -1094 + 7347 \times 299 \\
    1 &= 7347 \times 1393 + 9355 \times -1094 \\
    1 &= 9355 \times -6666 + 44767 \times 1393 \\
    1 &= 44767 \times 317809609 + 2134320847 \times -6666 \\
    1 &= 2134320847 \times -6666 + 44767 \times 317809609 \\
    \text{Found inverse -6666} \\
    y_3 &= -6666 \\
    s &= 30773171189753 \\
    x &= 31337
\end{align*}

The Chinese Remainder Theorem gives the unique solution (equivalence class)
for our system. Thus we can be confident in this solution for $s$.
The cube root of our solution was done in the expected way.
We recognise there will be other solution for the cube root but the definition
of the RSA encoding states our plaintext must be in the domain less than any
of our modulo's thus $x < min(n1,n2,n3)$.

Thus our final plaintext is $\sqrt[3]{s} = 31337$.

\section{Linear feedback shift}
Our recurrence relation is,
\begin{align*}
    l_{i+5} &= l_{i+4} + l_{i+3} + l_{i+2} + l_{i+1} + l_{i}
\end{align*}

\subsection{Prove $l_j+l_{j+2}+l_{j+4}$ is constant no matter $j$}
We shall prove this through induction. First consider the base case.
\begin{align*}
    l_0 + l_2 + l_4 &= l_1 + l_3 + l_5 \\
    l_0 + l_2 + l_4 &= l_1 + l_3 + l_4 + l_3 + l_2 + l_1 + l_0 \\
    l_0 + l_2 + l_4 &= l_4 + l_2 + l_0
    && \text{Addition and subtraction are equal in binary} \\
    \text{LHS} &= \text{RHS}
\end{align*}

Let $L_j = l_{j+4} + l_{j+2} + l_{j}$ and 
$L_{j+1} = l_{j+5} + l_{j+3} + l_{j+1}$

Now we assume for all $j \leq k$ we have $L_j = L_k$

Finally we check that $L_{k+1} = L_k$.
Observe,
\begin{align*}
    l_{k+5} &= l_{k+4} + l_{k+3} + l_{k+2} + l_{k+1} + l_{k} \\
    l_{k+5} - l_{k+3} - l_{k+1} &= l_{k+4} + l_{k+2} + l_{k} \\
    l_{k+5} + l_{k+3} + l_{k+1} &= l_{k+4} + l_{k+2} + l_{k} \\
    && \text{Addition and subtraction are equal in binary} \\
    L_{k+1} &= L_{k}
\end{align*}

Thus we have proven $l_j+l_{j+2}+l_{j+4}$ is constant no matter $j$.

\subsection{Show we only have $2^4$ distinct values}
We know that,
\begin{align*}
    l_{j+5} &= l_{j+4} + l_{j+3} + l_{j+2} + l_{j+1} + l_{j} \\
    l_{j+5} + l_{j+3} + l_{j+1} &= l_{j+4} + l_{j+2} + l_{j} \\
\end{align*}

We choose some sequence from $i$ to $i+4$ of the keystream. 
\begin{equation}
    l_{i+4} l_{i+3} l_{i+2} l_{i+1} l_{i} \\
\end{equation}

We know $l_{i+4} + l_{i+2} + l_{i} = c$ is a constant that is determined by
$l_{4} + l_{2} + l_{0}$. Thus
\begin{align*}
    l_{i+4} &= l_{i+2} + l_{i} + c && \text{Bitwise subtraction is addition} \\
\end{align*}

Thus we have at most 4 free bits as the final bit is determined by
$l_{i+2} + l_{i}$. From this it follows we have $2^4$ distinct 5 bit sequences.

\subsection{Period is at most $2^4$}
Consider a bit in the keystream $l_i$.

From here we observe that
\begin{align*}
    l_{i+6} + l_{i+4} + l_{i+2} &= l_{i+4} + l_{i+2} + l_{i} \\
    l_{i+6} &= l_{i} \\
\end{align*}

Since this is true for all $i$, we determine our period is at most 6 which
is less than $2^4$.

\end{document}
