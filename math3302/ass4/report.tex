\documentclass{article}
\usepackage{mathtools}
\usepackage{amsmath}
\usepackage{amssymb}
\usepackage{amsfonts}
\usepackage{graphicx}
\usepackage{float}
\usepackage{verbatim}

\linespread{1.3}
\setlength{\parindent}{0em}
\setlength{\parskip}{1em}
\setcounter{secnumdepth}{0}
\setcounter{MaxMatrixCols}{20}

\title{Assignment 4}
\author{Joshua Hwang (44302650)}
\date{25 May}

\begin{document}
\maketitle

\section{ENIGMA}
\subsection{Keyspace}
Each rotor of the ENIGMA can have 26 possible starting positions
(An English letter must be chosen). Thus the rotors have $26^3$
possible states.

The plugboard can have up to 6 wires.
Consider the case with all 6 wires used. We pick the first two letters out of
${}^{26}C_2$ then ${}^{24}C_2$ and so on for each wire giving us.
\[
    {}^{26}C_2 \times {}^{24}C_2 \times {}^{22}C_2 \times {}^{20}C_2
        \times {}^{18}C_2 \times {}^{16}C_2
\]

Since there is no order to the pairs we divide by $6!$. We repeat this
process for less than 6 wires. Our results are below,
\begin{align*}
    \text{6 wires} &= 100391791500 \\
    \text{5 wires} &= 5019589575 \\
    \text{4 wires} &= 164038875 \\
    \text{3 wires} &= 3453450 \\
    \text{2 wires} &= 44850 \\
    \text{1 wire}  &= 325 \\
\end{align*}

From here we sum them up to give 105578918575 for wire combinations.
Thus the ENIGMA keyspace is $105578918575 \times 26^3 1855655072874200$.
In scientific notation this is $1.86 \times 10^15$

\subsection{Knowing rotor products}

First looking at $A$ and $D$ we can see $A$ may pair \texttt{r} wth anything
from \texttt{omwbldxvyiqe}. From this single pairing the remaining
\texttt{guthapznfcs} are given pairs. Thus $A$ has 12 total possiblities and
$D$ is dependent on whatever configuration $A$ is.

$B$ and $E$ have the same situation except there are two choices to be made.
Pairing \texttt{u} with something in \texttt{lxnqvckbjm} and pairing \texttt{d}
with something in \texttt{fz}. From these two choices we have
$10 \times 2 = 20$.

Using the exact same technique shown with $BE$ we get $9 \times 3 = 27$ for
$CF$.

\subsection{Find rotor configurations}
We know \texttt{qfoezw} must be 6 of the same letter. For the first encrypted
letter we have \texttt{q} which must have come from the letters
\texttt{rguthapznfcs}. The second letter \texttt{f} must have come from
\texttt{dr} and since the encrypted letter \texttt{q} and \texttt{f}
are the same plain text we can determine that the letter must be \texttt{r}.

Just from this pairing we can determine that $A$ maps \texttt{r} to \texttt{q}.
Now the rotors can be deduced,
\begin{verbatim}
-----------------------------------For A---------------------------------------
(r, q)
(g, i)
(u, y)
(t, v)
(h, x)
(a, d)
(p, l)
(z, b)
(n, w)
(f, m)
(c, o)
(s, e)

(j, k)

-----------------------------------For D---------------------------------------
(q, g)
(i, u)
(y, t)
(v, h)
(x, a)
(d, p)
(l, z)
(b, n)
(w, f)
(m, c)
(o, s)
(e, r)

(j, k)
\end{verbatim}

Now we can easily determine that \texttt{h} maps to \texttt{x} for rotor $A$.
So we also know $B$ will map \texttt{x} to \texttt{o}.
We know $B$ maps \texttt{r} to \texttt{f} which allows us to determine
\texttt{(r, f)} and \texttt{(d, z)} pairs.

We also know \texttt{x} to \texttt{o} which gives us complete knowledge of
rotors $B$ and $E$.
\begin{verbatim}
-----------------------------------For B---------------------------------------
(r, f)
(d, z)

(o, x)
(w, l)
(a, m)
(h, j)
(t, b)
(g, k)
(p, c)
(u, v)
(i, q)
(y, n)

(e, s)

-----------------------------------For E---------------------------------------
(f, d)
(z, r)

(x, w)
(l, a)
(m, h)
(j, t)
(b, g)
(k, p)
(c, u)
(v, i)
(q, y)
(n, o)

(e, s)
\end{verbatim}

The process is repeated again for $CF$.
\begin{verbatim}
-----------------------------------For C---------------------------------------
(x, m)
(y, h)
(b, s)

(o, r)
(w, v)
(a, c)
(e, l)
(u, z)
(d, k)
(j, n)
(i, g)
(p, t)

(q, f)

-----------------------------------For F---------------------------------------
(m, y)
(h, b)
(s, x)

(r, w)
(v, a)
(c, e)
(l, u)
(z, d)
(k, j)
(n, i)
(g, p)
(t, o)

(q, f)
\end{verbatim}

\section{Substitution-permutation network}
The operation to create a differential table is quite straightforward,
\begin{align*}
    x^* &= x \oplus \Delta x \\
    y^* &= S(x^*) \\
    \Delta y &= y^* \oplus y \\
\end{align*}

Where the function $S(x)$ is the S-box.
The above process constructed the table below,
\begin{verbatim}
           |          delta Y          |
  X  |  Y  | delta X = 1 | delta X = 9 |
-----+-----+-------------+-------------+
  0  |  e  |      c      |      a      |
  1  |  2  |      c      |      d      |
  2  |  1  |      2      |      8      |
  3  |  3  |      2      |      6      |
  4  |  d  |      4      |      1      |
  5  |  9  |      4      |      1      |
  6  |  0  |      6      |      b      |
  7  |  6  |      6      |      1      |
  8  |  f  |      b      |      d      |
  9  |  4  |      b      |      a      |
  a  |  5  |      c      |      6      |
  b  |  9  |      c      |      8      |
  c  |  8  |      4      |      1      |
  d  |  c  |      4      |      1      |
  e  |  7  |      c      |      1      |
  f  |  b  |      c      |      b      |

\end{verbatim}

\subsection{Find specific differential trail}
ZZZ
1001 0000 0000 1001

0001 0000 0000 0001 sbox
0000 0000 0000 1001 permute

0000 0000 0000 0001 sbox
0000 0000 0000 0001 permute

0000 0000 0000 0010 sbox
0000 0001 0000 0000 permute

\subsection{Determine which bits of $K_5$ can be deduced}
The bits from $K_{5,5} \to K_{5,8}$.


\end{document}
